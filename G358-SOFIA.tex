%                                                                 aa.dem
% AA vers. 9.1, LaTeX class for Astronomy & Astrophysics
% demonstration file
%                                                       (c) EDP Sciences
%-----------------------------------------------------------------------
%
%\documentclass[referee]{aa} % for a referee version
%\documentclass[onecolumn]{aa} % for a paper on 1 column  
%\documentclass[longauth]{aa} % for the long lists of affiliations 
%\documentclass[letter]{aa} % for the letters 
%\documentclass[bibyear]{aa} % if the references are not structured 
%                              according to the author-year natbib style

%
%\documentclass[longauth,usenatbib]{aa}  
\usepackage{natbib,twoopt}
\usepackage{threeparttable}
\usepackage{adjustbox}
\usepackage{rotating}
% for \nobreakdash
\usepackage{amsmath}

%
\usepackage{graphicx}
%%%%%%%%%%%%%%%%%%%%%%%%%%%%%%%%%%%%%%%%
\usepackage{txfonts}
%%%%%%%%%%%%%%%%%%%%%%%%%%%%%%%%%%%%%%%%



% To add links in your PDF file, use the package "hyperref"
% with options according to your LaTeX or PDFLaTeX drivers.
\usepackage[breaklinks=true,colorlinks=true,allcolors=blue,pagebackref=true]{hyperref}

\newcommand{\mypm}{\mathbin{\smash{%
\raisebox{0.65ex}{%
            $\underset{\raisebox{0.05ex}{$\smash -$}}{\smash+}$%
            }%
        }%
    }%
}

\newcommand{\goodchi}{\protect\raisebox{2pt}{$\chi$}}
\newcommand*\red{\color{red}}

\begin{document} 



   \title{Infrared observations of the flaring maser source G358.93$-$0.03\thanks{Based on observations collected at the European Organisation for Astronomical Research in the Southern Hemisphere under ESO programme 0103.C-9033(A). Based on observations made with the NASA/DLR Stratospheric Observatory for Infrared Astronomy (SOFIA) under Proposal IDs 75\_0037 and 08\_0163.}}
   \subtitle{SOFIA confirms an accretion burst from a massive young stellar object
   }

   \author{B. Stecklum\inst{1}
          \and
          V. Wolf\inst{1}
          \and
          H. Linz\inst{2}
          \and
          A. Caratti o Garatti\inst{3}
          \and
          S. Schmidl\inst{1}
          \and
          S. Klose\inst{1}
          \and
          J. Eislöffel\inst{1}
          \and 
          Ch. Fischer\inst{4}
          \and
          C. Brogan\inst{5}
          \and
          R. Burns\inst{6}
          \and
          O. Bayandina\inst{7,8}
          \and
          C. Cyganowski\inst{9}
          \and
          M. Gurwell\inst{10}
          \and
          T. Hunter\inst{4}
          \and
          N. Hirano\inst{11}
          \and
          K.-T. Kim\inst{12}
          \and
          G. MacLeod\inst{13}
          \and
          K. M. Menten\inst{14}
          \and
          M. Olech\inst{15}
          \and
          G. Orosz\inst{16}
          \and
          A. Sobolev\inst{17}
          \and
          T. K. Sridharan\inst{10}
          \and
          G. Surcis\inst{18}
          \and
          K. Sugiyama\inst{19}
          \and
          J. van der Walt\inst{20}
          \and
          A. Volvach\inst{21}
          \and
          Y. Yonekura\inst{22}
%          \fnmsep\thanks{Just to show the usage of the elements in the author field}
          }

   \institute{
            Thüringer Landessternwarte Tautenburg, Sternwarte 5, 07778 Tautenburg, Germany,
             \email{stecklum@tls-tautenburg.de}
        \and
            Max Planck Institute for Astronomy, Königstuhl 17, 69117 Heidelberg, Germany
        \and
            Dublin Institute for Advanced Studies, 31 Fitzwilliam Place, D02 XF86, Dublin, Ireland
        \and
            Deutsches SOFIA Institut, University of Stuttgart, 70569 Stuttgart, Germany
        \and
            National Radio Astronomy Observatory, 520 Edgemont Road, Charlottesville, VA 22903, USA
        \and
            Mizusawa VLBI Observatory, National Astronomical Observatory of Japan, Osawa 2-21-1, Mitaka, Tokyo 181-8588, Japan ; Korea Astronomy and Space Science Institute, 776 Daedeokdae-ro, Yuseong-gu, Daejeon 34055, Republic of Korea
        \and
            Joint Institute for VLBI ERIC, Oude Hoogeveensedijk 4, 7991 PD Dwingeloo, The Netherlands
         \and
            Astro Space Center, P.N. Lebedev Physical Institute of RAS, 84/32 Profsoyuznaya st., Moscow, 117997, Russia
        \and
            SUPA, School of Physics and Astronomy, University of St. Andrews, North Haugh, St. Andrews KY16 9SS, UK
        \and
            Center for Astrophysics Harvard | Smithsonian, Cambridge, MA 02138, USA
        \and
            Academia Sinica Inst of Astronomy \& Astrophysics ASIAA PO Box 23-141. Taipei 106. China
        \and
            Korea Astronomy and Space Science Institute, 776 Daedeokdae-ro, Yuseong-gu, Daejeon, 34055, Republic of Korea
        \and
            Hartebeesthoek Radio Astronomy Observatory, P.O. Box 443, Krugersdorp 1740, South Africa ; The University of Western Ontario, 1151 Richmond Street, London, ON N6A 3K7, Canada
        \and
            Max-Plank-Institut für Radioastronomie, Auf dem Hügel 69, 53121 Bonn, Germany
        \and
            Institute of Astronomy, Faculty of Physics, Astronomy and Informatics, Nicolaus Copernicus University, Grudziadzka 5, 87-100 Torun, Poland
        \and
            Xinjiang Astronomical Observatory, Chinese Academy of Sciences, Urumqi, Xinjiang, China
        \and
            Astronomical Observatory, Institute for Natural Sciences and Mathematics, Ural Federal University, 19 Mira street, Ekaterinburg 620002, Russia)
        \and
            INAF-Osservatorio Astronomico di Cagliari, Via della Scienza 5, 09047, Selargius, CA, Italy
        \and
            Mizusawa VLBI Observatory, National Astronomical Observatory of Japan (NAOJ), 2-21-1 Osawa, Mitaka, Tokyo 181-8588, Japan
        \and
           Space Research Unit, Physics Department, North West University, Potchefstroom 2520, South Africa; Department of Physics and Astronomy, Faculty of Physical Sciences, University of Nigeria, Carver Building, 1 University Road, 410001, Nsukka, Nigeria
        \and
            Radio Astronomy Laboratory of Crimean Astrophysical Observatory RAS, Katsively, RT-22 Crimea
        \and
            Center for Astronomy, Ibaraki University, 2-1-1 Bunkyo, Mito, Ibaraki 310-8512, Japan
        }

   \date{Received October 11, 2020; accepted YYYY ZZ, 2020}

% \abstract{}{}{}{}{} 
% 5 {} token are mandatory
 
  \abstract
  % context heading (optional)
    {
   Class~II methanol masers are signposts of massive young stellar objects (MYSOs). Recent evidence shows that flares of these masers are driven by MYSO accretion bursts. Thus, maser monitoring can be used to identify such bursts which are hard to discover otherwise. Infrared observations reveal burst-induced changes in the spectral energy distribution, first and foremost a luminosity increase, which provide valuable information on a very intense phase of high-mass star formation.
   }
  % aims heading (mandatory)
   {
   In mid-January 2019, flaring of the 6.7\,GHz CH$_3$OH maser (maser for short) of the MYSO G358.931-0.030 (G358 for short) was reported. The international maser community (M2O) initiated an extensive observational campaign  which revealed extraordinary maser activity and yielded the detection of numerous new masering transitions. ALMA/SMA sub-arcsecond (sub)millimeter imaging resolved the maser emitting core of the  star forming region and proved the association of the masers with the brightest continuum source (MM1), which hosts a hot molecular core. These observations, however, failed to detect a significant rise in the (sub)millimeter dust continuum emission. Therefore, we performed near-infrared (NIR) and far-infrared (FIR) observations to prove or disprove whether the CH$_3$OH flare was driven by an accretion burst. 
   %While the radiative excitation of Class~II methanol masers suggests that the flare is due to enhanced protostellar luminosity, direct evidence for an accretion burst was lacking. In order to settle this point we performed near- and far-infrared observations. We aim to derive major burst quantities from the change of the spectral energy distribution due to the burst.
   }
  % methods heading (mandatory)
   {
    NIR imaging with GROND has been acquired and far-infrared integral-field spectroscopy with FIFI-LS aboard SOFIA was carried out on two occasions to detect possible counterparts to the (sub)millimeter sources and compare their photometry to archival measurements. The comparison of pre-burst and burst spectral energy distributions is of crucial importance to judge whether a substantial luminosity increase, caused by an accretion burst, is present and
    %gave rise to the particular maser activity
    triggered the maser flare. Radiative transfer modeling of the 
    %pre- and burst 
    spectral energy distribution (SED) of the dust continuum emission at multiple epochs provides valuable information on the bursting MYSO.
   }
  % results heading (mandatory)
   {
   %GROND $JHK_s$ imaging showed a very red object associated with G358 which, however, is unrelated to the burst since it did not brighten considerably and does not coincide with MM1.
   %This confirms the deeply embedded nature of the latter. 
   The far-infrared fluxes of MM1 measured with FIFI-LS exceed those from %HI-GAL 
   {\em Herschel} significantly which clearly confirms the presence of an accretion burst. The second epoch data, taken 
   %$\approx$
   about 16 months later, still show increased fluxes. Our radiative transfer modeling yielded major burst parameters and suggests that the MYSO
   %is deeply embedded and 
   features a circumstellar disk which might be transient.
   %{\red 
   From the pre-, burst and post-burst SEDs, conclusions on heating and cooling time-scales could be drawn. Circumstances of the burst-induced maser relocation have been explored.
   %}
   %and y major burst parameters are {\tt\red L increase, accretion rate/mass}.
  % While MM1 went undetected in the GROND imaging the very red NIR counterpart of MM3 was identified.
   }
  % conclusions heading (optional), leave it empty if necessary 
   {
   The verification of the accretion burst from G358 is another confirmation that Class~II methanol maser flares represent an alert for such events. Thus, monitoring of these masers greatly enhances the chances to identify MYSOs during periods of intense growth. 
   %This suggests that a sample of MYSO burst events could be established within a reasonable amount of time, allowing a comparison with predictions from models of high-mass star formation. 
   The few events known to date already indicate that there is a broad range in burst strength and duration as well as environmental characteristics. The G358 event is the shortest and least luminous  accretion burst known to date. According to models, bursts of this kind occur most often.
   % and 
   % was, nevertheless, accompanied by
   % featured the most diverse maser activity. 
   %This might be related to the extreme youth of the MYSO.
   }

   \keywords{Accretion, accretion disks -- Stars: formation --
                Stars: protostars --
                Stars: individual objects (\object{G358.93-0.03}) -- Radiative transfer
               }

   \maketitle
%
%-------------------------------------------------------------------

\section{Introduction}\label{intro}
The collapse of dense molecular cloud cores gives rise to the birth of stars, a process which was thought to proceed in a smooth and continuous fashion. However, first evidence for the unsteady growth of forming stars emerged by recognizing that the outburst of FU Orionis, thought to be a rare phase of early stellar evolution \citep{1966VA......8..109H}, was instead an episode of enhanced disk accretion \citet{1985ApJ...299..462H}. Since then it has been realized that episodic accretion is an intrinsic feature of forming young stars \citep{1996ARA&A..34..207H, 2014prpl.conf..387A}. While this knowledge had been established exclusively from observations of low-mass stars, which become optically visible while still accreting, it was unknown until recently whether high-mass stars ($M_{\star}{\gtrsim}8\,{\rm M}_\odot$) show the same behavior during their formation. 
%Actually, t
%{\red
Their scarcity and fast formation timescales %,
%%short-livedness
%which 
imply that they are still deeply embedded in their parental core while reaching the main sequence. This suggests that similar outbursts during high-mass star formation, if present at all, might be difficult to detect. 
%}

A candidate young massive eruptive variable, V723 Car, was identified by \citet{2015MNRAS.446.4088T}. The object brightened in the $K$ band by 3.7\,mag between 1993 and 2003. Since the outburst was found a posteriori no information on the possible accretion luminosity is available. The post-burst luminosities range from $2.5\,{\times}\,10^3\,{\rm L}_\odot$ \citep{2011ApJS..194...14P} to ${\approx}\,4\,{\times}\,10^3\,{\rm L}_\odot$ \citep{2015MNRAS.446.4088T} which correspond to a mass of 8 -- 9\,${\rm M}_\odot$ \citep{1996MNRAS.281..257T} for a Zero-Age-Main Sequence (ZAMS) object.
%{\red 
So, while the V723 Car event might be considered to be a possible MYSO accretion burst, the lack of observational coverage before and at the time of its incidence precludes major conclusions with regard to high-mass star formation.
%}

Recently the situation concerning MYSO accretion bursts changed all of a sudden by the discoveries of the almost coincident events from the MYSOs S255IR-NIRS3 \citep{2016ATel.8732....1S, 2017NatPh..13..276C, 2018ApJ...863L..12L} and NGC6334I-MM1 \citep{2017ApJ...837L..29H, 2018ApJ...854..170H}. The luminosity increase, seen at infrared (IR) and (sub)mm wavelengths for S255IR-NIRS3 and
% solely 
in the (sub)mm regime in the case of NGC6334I-MM1, provided direct evidence for enhanced accretion rates. Most notably, these outbursts were accompanied by flares of Class~II methanol masers (methanol masers for short; \citealp{2015ATel.8286....1F,2018A&A...617A..80S, 2018MNRAS.478.1077M}). This confirmed a radiative pumping mechanism of this kind of methanol masers \citep{1991ASPC...16..119M, 1997A&A...324..211S, 2005MNRAS.360..533C}, which is consistent with variability studies of the maser emission \citep{2018MNRAS.474..219S, 2019MNRAS.485..777D}. Since methanol masers trace the very early stages of massive star formation (e.g., \citealp{2013MNRAS.435..524B}), maser flares might be taken as a proxy for accretion variability of the protostellar host. Keeping this in mind the international maser community established the Maser Monitoring Organization (M2O)\footnote{See M2O website at \url{http://MaserMonitoring.org}} to coordinate single-dish monitoring of masers and interferometric follow-up measurements.

G358 (RA: $\rm 17^h 43^m 10
%\overset{s}{.}{02}
.\!^s02{}$, $\rm {DEC\!:~} {-}29\degr 51' 45\farcs8$, J2000) represents a hitherto little explored massive star forming site as evident from just eight SIMBAD entries until 2018. A kinematic distance estimate of 6.75$\,{\mypm}\,^{\,0.37}_{\,0.68}$\,kpc was derived by \citet{2019ApJ...881L..39B}. It is consistent with GAIA distances of visible stars in the G358 region of $\lesssim$5\,kpc
%{\red 
which impose a lower limit to the distance of the molecular cloud hosting G358
%}
\citep{2020NatAs...4..506B}.
In mid-January 2019, flaring of the 6.7\,GHz CH$_3$OH maser line \citep{1991ApJ...380L..75M} in G358 was announced \citep{2019ATel12446....1S}. Thus, for the first time, M2O orchestrated an extensive observing campaign which became extremely successful. 

Thanks to the immediate response, an unprecedented wealth of masering lines, including numerous new transitions, could be observed \citep{2019ApJ...876L..25B, 2019ApJ...881L..39B, 2019MNRAS.489.3981M} and new maser species were discovered \citep{2020ApJ...890L..22C, 2020NatAs.tmp..144C}. ALMA/SMA (sub)millimeter imaging dissected the star forming region and pinpointed the MYSO which hosts the flaring masers \citep{2019ApJ...881L..39B}. In all likelihood, the brightest continuum source MM1, which turned out to be a hot molecular core, experienced an accretion burst. For the first time, a spectacular confirmation of the event was achieved by high-resolution, multi-epoch observations of the 6.7-GHz methanol maser emission which revealed outward maser spot propagation, tracing the spread of the thermal radiation emanating from the burst \citep{2020NatAs...4..506B}.
However, without evidence for a significant rise in (sub)millimeter dust continuum emission from
%any of the sources 
MM1 \citep{2019ApJ...881L..39B}, the energetics of the burst remained an open issue. Therefore, we aimed for observations to identify the IR counterpart of MM1 and to verify its luminosity increase, thus independently confirming the third MYSO accretion burst witnessed so far. 

At present, the Stratospheric Observatory for Infrared Astronomy (SOFIA, \citealp{1993AdSpR..13..549E, 2012ApJ...749L..17Y}) is the only facility which offers the capability to trace the far-infrared (FIR) flux increase caused by such an event. Consequently, an attempt was made for observing G358 with SOFIA which turned out to be successful. The observational results in context with supplementary data, the analysis and interpretation are the subjects of the present paper and will be outlined in the following.

The paper is organized as follows. At the beginning the observational foundation will be explained. The next part deals with deriving constraints on the spectral energy distribution of the bursting source. The estimation of the luminosity increase, the central quantity for assessing the accretion burst, is performed in two steps. First, a simplified treatment using graybody functions is applied. Then, a more thorough analysis is performed, utilizing dust continuum radiative transfer. The discussion section concludes the paper in which results of the present investigation are put in context with regard to previous observational and theoretical findings.

We note that the term {\em luminosity}, if not declared otherwise, refers to the bolometric luminosity throughout the paper. Similarly, the term or value {\em error} always implies the 1$\sigma$ error/standard deviation unless noted otherwise. Magnitudes are based on the Vega system.


%--------------------------------------------------------------
\section{Observations}\label{obs}
\subsection{Archival data}\label{arch}
Archival data is essential for establishing the presence of an accretion burst since it constrains the source luminosity during the pre-burst state. Because G358 its located in the Galactic center region it has been covered by a wealth of surveys. For the present study IR fluxes and positions have been compiled from the 2MASS \citep{2003yCat.2246....0C},
VVV \citep{2010NewA...15..433M},
ISOGAL \citep{2003A&A...403..975O},
GLIMPSE \citep{2009yCat.2293....0S},
(NEO)WISE \citep{2014ApJ...792...30M},
MIPSGAL \citep{2015AJ....149...64G} and
ATLASGAL \citep{2009A&A...504..415S} surveys. 
% HI-GAL
%\cite{2016yCat..35910149M},
%The HI-GAL fluxes were color corrected according to Table \,5.8 of the SPIRE handbook for a dust temperature of 25\,K and a dust emissivity index of 2.0 (see Sec.\,\ref{gb}).
The comparison of the HI-GAL photometry \citep{2016yCat..35910149M} with that performed by the ATLASGAL team \citep{2013A&A...549A..45C} revealed
%substantial differences
%{\red 
inconsistencies
%}
at longer wavelengths and with regard to the error estimates. Therefore, we
% measured the {\em Herschel} fluxes
performed photometry on our own on the corresponding PACS and SPIRE images  (epoch 2010 September 07), retrieved from the NASA/IPAC Infrared Science Archive (IRSA)\footnote{\url{https://irsa.ipac.caltech.edu}}, using the MPFit2DPeak function from the IDL Astronomy library \citep{1995ASPC...77..437L}. The point spread function was chosen to be a 2D Gaussian which yielded the best fit over other representations (Moffat, Lorentzian profiles). The background was estimated on a fixed annulus for all wavelengths to ensure consistency of sky estimates.
%retrieved from the ATLASGAL database server (\cite{2013A&A...549A..45C}).

The target is not included in the AKARI Bright Source Catalog \citep{2010yCat.2298....0Y} but can be identified in Far-Infrared Surveyor images (FIS, \citealp{2015PASJ...67...50D}) taken with the N60 and WIDE-S filters (epoch
%FITS header entry     
%DATE-OBS= '2000-01-01T12:00:00' / J2000 equinox  
2007 January). Correspondingly, photometry was performed on those frames in the same way as described above. Since the WIDE-S image suffers from severe striping, the resulting flux has a substantial uncertainty and will therefore be
% considered as lower limit
neglected.

%{\tt\red compilation of SED, AKARI FIS photometry, VVV}

\subsection{Near-infrared imaging}\label{nir}
Optical and near-infrared (NIR) imaging of G358 was performed using the seven-channel Gamma-Ray Burst Optical/Near-infrared Detector GROND \citep{2008PASP..120..405G}, using director's discretionary time (DDT) at the MPG/ESO 2.2\,m telescope at La Silla (Chile) on 2019 February 8. GROND obtains images in seven bands (optical: Sloan {\it g'\,r'\,i'\,z'}, NIR: $J H K_{\rm s}$) simultaneously. The total integration time amounts to 38 minutes. Data processing was performed by means of the GROND pipeline \citep{2008ApJ...685..376K}.

\subsection{Far-infrared integral-field spectroscopy}\label{fifi}
The Field-Imaging Far-Infrared Line Spectrometer (FIFI-LS, \citealp{2000SPIE.4014...14L, 2018JAI.....740003F, 2018JAI.....740004C}) is a far-infrared integral field spectrograph aboard SOFIA. FIFI-LS features a blue and red channel in parallel which provide an overall wavelength coverage from 51\,$\mu$m to 203\,$\mu$m. This matches very well the range of the spectral energy distribution (SED), where MYSOs emit the bulk of their energy by dust continuum radiation and where the relative flux increase, due to an accretion burst, is highest \citep{2019MNRAS.487.4465M}. Thus, instruments like FIFI-LS provide the best prospects to detect the luminosity increase due to accretion bursts. 

A DDT proposal to perform spectro-photometry of the FIR dust emission of G358 using FIFI-LS was submitted in mid-February 2019.
%with the expectation to catch the IR emission while it is still on the rise. 
One hour of observing time was awarded 
%for the project 75\_0037 
by the Director of SOFIA Science Mission Operations. The measurements were performed on 2019 May 1 when the maser flare was still strong but already decaying (MacLeod et al., in prep.; Yonekura et al., in prep.). Several spectral bands (see Table \,\ref{fifiphot}) were chosen to sample the full spectral range of FIFI-LS and, at the same time, cover high rotational transitions of the CO molecule. The spectral scan length per sub-band ranged from 0.3 to 1.0\,$\mu$m.
%The spectral resolution of the instrument amounts to R{$\approx$}500-2000, with an instantaneous coverage between 800\,km/s and 3000\,km/s. 
A regular follow-up proposal for SOFIA Cycle 8 was submitted and accepted as well. 
%However, measurements could not be performed yet 
Due to the flight suspension induced by the COVID-19 pandemic, the observations were delayed and eventually %\textcolor{red}{why eventually?}
% "eventually" -> schließlich, endlich; kann man reinfallen
performed on 2020 August 28. The same settings and amount of observing time as for the first epoch measurements were used. However, because of a technical issue, no data could be obtained in the blue channel. Yet, 
% us observations at two epoch allow us 
drawing conclusions on the flux evolution from the two epochs became possible.
% could not be done at. 
%While the single epoch data allow us to assess the presence of a burst, no conclusion on its post-burst FIR variability can be made.


\section{Data analysis and results}\label{res}

As evidenced by the ALMA/SMA observations, G358 is a star forming complex harboring several massive protostars \citep{2019ApJ...881L..39B}. Because of its considerable distance and the large beam sizes of observing facilities at mid-infrared (MIR) and FIR wavelengths the photometry obtained by the latter represents the total flux. The following 
%criteria will be applied to 
considerations aim at identifying and characterizing the IR counterpart of MM1. This includes an approach to account for flux contributions of all other components. Eventually, conclusions on the nature of MM1 will be drawn based on radiative transfer (RT) modeling of its pre-burst and burst SEDs.

\subsection{NIR imaging and source identification}\label{rnir}
%{\red 
As shown in Fig.\,\ref{fig:GROND},
%}
a NIR source is situated close to the position of G358. It is listed as 2MASS~J17431001$-$2951460 in the 2MASS All-Sky Catalog of Point Sources \citep{2003yCat.2246....0C}.
While 2MASS detected it only in the $K_{\rm s}$ band at 11.5$\,{\pm}\,$0.07\,mag, the deeper {\bf V}ISTA {\bf V}ariables in {\bf V}ia Lactea Survey (VVV, \citealp{2017yCat.2348....0M}) yielded an H-band detection as well. The color index of $H-K_{\rm s}$\,=\,3.66\,mag indicates a very red object. Our GROND imaging detected it in the $J, H$ and $K_{\rm s}$ bands, but not at shorter wavelengths. In the VVV survey, the object has shown rapid brightness changes within a 3$\sigma$ range of 0.4\,mag, and a peak-to-peak variation of 0.79\,mag during five years of VVV $K_{\rm s}$-band monitoring. 

The GROND photometry indicated a $K_{\rm s}$ brightness increase by 0.34\,mag with respect to the mean of 12.23\,mag, i.e., within the range of common YSO variability. Its position, within 0\farcs2, is consistent with the secondary hot molecular core and dust continuum source MM3 detected by ALMA, which is located 1\farcs09 to the southwest of the main hot molecular core MM1 \citep[cf.][]{2019ApJ...881L..39B}. Therefore, the NIR source is likely the IR counterpart of MM3 and, thus, unrelated to the outburst. The VVV and GROND NIR color composite images are shown for comparison in Fig.\,\ref{fig:GROND}.

The $K_{\rm s}$ image of the difference GROND$-$VVV, obtained after flux scaling and convolving to the same spatial resolution, does not provide evidence for the presence of a light echo from an accretion outburst, unlike for the case of S255IR-NIRS3 \citep{2017NatPh..13..276C}. This may be another sign for the high extinction toward the bursting source.

\begin{figure}   % Fig 1
\centering
%	\includegraphics[width=\columnwidth]{vg_3.png}
	\includegraphics[width=\columnwidth]{vg_4.png}
	\caption{\textbf{top:} VVV $JHK_s$ color composite of the G358 region (epoch 2010 August 15). The positions of MM1 and MM3 from \citet{2019ApJ...881L..39B} are marked. Black plus signs denote positions from IR observations at wavelengths up to 24\,$\mu$m. \textbf{bottom:} GROND $JHK_s$ color composite (epoch 2019 February 8) with contours of the ALMA 0.89\,mm continuum map \citep{2019ApJ...881L..39B}.
	}
 \label{fig:GROND}
\end{figure}


\subsection{WISE/(NEO)WISE photometry}\label{rneo}

\begin{figure*}
	\sidecaption
%	\includegraphics[width=\columnwidth]{G358_W1_W2.png}
	\includegraphics[width=12cm]{G358_W1_W2.png}
	\caption{(NEO)WISE W1 (blue) and W2 (red) light curves based on mean magnitudes and respective errors for each visit. The first two epochs are from the WISE mission. Vertical lines mark the dates of the flare peak (black) and the first FIFI-LS epoch (green). Horizontal lines indicate mean magnitudes (solid) and their errors (dashed).
	}
 \label{fig:NW_lc}
\end{figure*}

Due to its orbit, the WISE IR space telescope \citep{2010AJ....140.1868W} visits a region in the sky twice a year. For G358 the 2019 visits occurred on March 17 and August 27. The first one almost coincided with the peak of the maser flare (MacLeod et al., in prep.; Yonekura et al., in prep.) which maximized chances for a possible mid-IR (MIR) detection.
%would have been extremely interesting. 
%Immediate information can be retrieved from
Photometry for G358 from the WISE and subsequent (NEO)WISE \citep{2014ApJ...792...30M} missions were retrieved from IRSA, covering observations until end of 2019. A saturation correction has been applied\footnote{See\,\,\url{http://wise2.ipac.caltech.edu/docs/release/neowise/expsup/sec2\_1civa.html}} to account for a photometric bias.

The (NEO)WISE W1 (3.4\,$\mu$m) and W2 (4.6\,$\mu$m) light curves are shown in Fig.\,\ref{fig:NW_lc}. 
%The last but one value for each band was taken close to the peak of the maser flare
%which occurred around 2019 April 1 \citep{2020NatAs.tmp..144C}. 
Because of its brightness in both filters bands, G358 led to small (W1) or mild (W2) detector saturation. In this case, the derivation of the brightness from the wings of the point spread function (c.f. section IV.4.c.iii of the WISE All-Sky Explanatory Supplement, \citealp{2012wise.rept....1C}) leads to enhanced scatter. 
% This is indicated by the the  mutual correlation to a value of 0.57.
Nevertheless, the discovery of an 
%reasonable 
obvious brightening which usually accompanies enhanced accretion should have been possible. However, there is no clear-cut evidence in both light curves for a flux increase at the burst epoch or later on.
From the scatter of the photometric values, a possible increase due to the burst can be constrained. Assuming a 2$\sigma$ detection limit for W1 and W2 of $\approx$0.5\,mag, i.e., a joint 2.8$\sigma$ limit, upper bounds for a possible flux contribution caused by the burst can be derived as 0.10\,Jy and 0.45\,Jy, respectively. 
% http://morpheus.phys.lsu.edu/cgi-bin/magnitudes.pl?magnitude=8.0151296517074098&wavelength=10.0&filter=WISE+W1&option=magnitude ->F_nu = 0.192604 Jy
% http://morpheus.phys.lsu.edu/cgi-bin/magnitudes.pl?magnitude=5.6830042201101856&wavelength=10.0&filter=WISE+W2&option=magnitude -> F_nu = 0.915775 Jy
The failure of the burst detection suggests that, at any time, MM3 provided by far most, if not all, emission seen in the (NEO)WISE bands. Further details on the (NEO)WISE astrometry are outlined in Sect.\,\ref{nw}.

\subsection{Infrared photo-astrometry}\label{iras}
In case of imaging an unresolved crowded region a shift of the emission centroid may be caused by differing object SEDs which would introduce a wavelength dependence and/or by variability leading to a temporal displacement. By placing upper limits on a possible centroid shift, constraints on the contribution of a single source with known position, MM1 in the present case, to the overall emission can be established. In the following this will be done using % (NEO)WISE as well as {\em Spitzer} MIPS 
the present infrared data.

\subsubsection{VVV and GROND}\label{vg}
An upper limit to the MM1 pre-burst 2.18\,$\mu$m flux can be derived from the stacked VVV $K_{\rm s}$ image. It is based on 131 exposures of 4\,s each, i.e., corresponds to a total integration time of 8.7\,min. Confusion noise due to the high surface density of objects in the Galactic center neighborhood limits the detection of a possible faint NIR counterpart of MM1. It has to be brighter than $K_{\rm s}\,{\approx}\,$15.5\,mag to be recognized next to 2MASS~J17431001$-$2951460. This corresponds to a flux density of 0.42\,mJy \citep{2003AJ....126.1090C}. Similarly, the burst $K_{\rm s}$ magnitude can be constrained by the corresponding GROND image. Given the aperture sizes of the 2.2-m and VISTA telescopes as well as the total integration times, the GROND image should almost reach the sensitivity of the stacked VVV frame. However, inferior seeing and slightly elliptical images reduce its depth by about half a magnitude, resulting in a burst 2.15\,$\mu$m upper limit of 0.66\,mJy.

\subsubsection{(NEO)WISE}\label{nw}

\begin{figure}
    \centering
	\includegraphics[width=\columnwidth]{G358_dRA_dDEC.png}
	\caption{(NEO)WISE coordinate offsets relative to an origin at MM1 (from \cite{2019ApJ...881L..39B}). Pre-burst, burst, and post-burst data are shown in blue, red and green. Thick error bars represent mean positions while individual appear thin. Positions of MM1 and MM3 (measured with ALMA) are marked by the diamond and the square, respectively. The arrow points towards the bright 2MASS~J17431001$-$2951460, which is at a distance of 9\farcs9 from the (0,0) position. The yellow circle indicates the W1 image FWHM of 6\farcs1.
	}
 \label{fig:NW_pos}
\end{figure}

For the following analysis (NEO)WISE positions were retrieved within 5\arcsec{} of the MM1 location  from IRSA, based on frames with the photometric quality flag A in both bands, a signal-to-noise ratio ${\ge}$\,20 and a frame quality rating of 10. The measurement quality and source reliability information flags of all frames, however, indicate contamination by the nearby 2MASS~J17430939$-$2951517 which is situated 9\farcs9 southwest of the MM3 and brighter in both (NEO)WISE bands.
This results from the image full width at half maximum (FWHM) of 6\farcs1 in the W1 band
\footnote{See\,\,\url{https://wise2.ipac.caltech.edu/docs/release/allwise/expsup/sec1_2.html}}.

The offsets of the individual positions of the IR source with regard to MM1 are shown in Fig.\,\ref{fig:NW_pos} where the data was divided into three groups. The pre-burst one includes all positions until the last visit before the burst (2018 August, blue), the burst group comprises those of the 2019 March visit (red) and the post-burst group represents the 2019 August visit (green). The quantitative analysis confirms the visual impression for the mean positions that both, burst and post-burst, are consistent with the pre-burst location, situated close to MM3. So most of the emission seen by (NEO)WISE arises from the MM3 NIR counterpart 2MASS~J17431001$-$2951460, with some contamination from the nearby IR source to the south-west which causes the elongated distribution of positions.
While the mean (NEO)WISE post-burst position (green) is in between MM1 and MM3, its large position error precludes drawing any conclusions from this fact on whether or not this is a late sign of the burst. These findings are consistent with the non-detection of the burst in the (NEO)WISE light curves (Sec.\,\ref{rneo}).

\subsubsection{{\em Spitzer} MIPS}\label{mips}
As indicated by Fig.\,\ref{fig:GROND} (top) the positions of the IR counterparts at wavelengths up to 24\,$\mu$m are almost coincident with ALMA source MM3. Supposing about equal flux densities of MM1 and MM3 at this wavelength, the centroid of the MIPSGAL image of G358 should be located halfway between both sources. This is clearly not the case. Thus we conclude that the pre-burst MIPS 24\,$\mu$m flux of MM1 was considerably smaller than the joint flux of MM1 and MM3 (4.58$\,{\pm}\,$0.02\,Jy, \citealp{2009ApJS..181..227H}). An upper flux limit for MM1 can be derived by assuming that it leads to a detectable centroid shift of three times the positional error of 0\farcs02 \citep{2009ApJS..181..227H}. 
%This would require 
%{\red
Since such a shift has not been detected, the contribution of MM1 to the total flux must be less than 0.42\,Jy. The upper limit is valuable to constrain the pre-burst luminosity.
%}

\subsection{SOFIA FIFI-LS spectro-photometry}\label{rfifi}
The FIFI-LS data was processed by the 'FIFI\_LS\_REDUX' pipeline  (version 1.7.0, \citealp{2015ASPC..495..355C}) and downloaded from the SOFIA Data Cycle System. The observations yielded a clear detection of continuum emission from the target in all bands. In several bands CO line emission has been detected as well which will be discussed elsewhere. 

The continuum fluxes were derived from an image consisting of the pixel-wise median of a spectral data cube along the wavelength dimension, thus free from emission-line flux. For both epochs, central wavelengths and derived flux densities together with an error measure are given in Table \,\ref{fifiphot}. We note that our observations revealed the need for an empirical correction of the 118.9\,$\mu$m flux. It was derived from SED fits of other MYSOs of our FIFI-LS data set which show a similar flux deficit as well in this band. This was also
confirmed by looking at data from the flux calibrators Mars and Uranus.

The measurement errors derived from the error images provided by the FIFI\_LS\_REDUX pipeline do not include the calibration uncertainty. Therefore, we adopted a conservative approach by using an uncertainty of 10\%, cf. \citet{2019AAS...23420805F}. Since a narrow dust emission feature at 69\,$\mu$m \citep{2013A&A...553A...5S} - not covered by our FIFI-LS bands - is the only one in this wavelength region, a low-order polynomial fit seems to be representative for the actual SED at the time of the observing epoch. The residuals from these fits listed in Table \,\ref{fifiphot}  correspond to a mean relative error which indeed equals the above uncertainty. 

With the SOFIA flux densities at hand the change of the SED due to the burst and its temporal evolution can be evaluated from
Fig.\,\ref{fig:SEDS} which shows, among others, the pre-burst SED (blue symbols), based on MIPS, AKARI, and
%HI-GAL
{\em Herschel}
data as well as the emission-line corrected ATLASGAL 870\,$\mu$m flux from \citet{2019ApJ...881L..39B}. The SED based on our first-epoch FIFI-LS observations (epoch 2019 May 1) and the ALMA 870\,$\mu$m integrated G358 flux from \citet{2019ApJ...881L..39B} (epoch 2019 April 12) is shown in red. The burst SED features flux densities larger than a factor of {$\gtrsim$}\,2 when compared to the pre-burst ones and a possible change of the SED shape. Thus, a luminosity increase, the prime signature of an accretion burst, has been witnessed with SOFIA. It represents the second confirmation of such an event
after the S255IR-NIRS3 burst \citep{2017NatPh..13..276C}
using this unique facility. Given the non-detection in the NIR/MIR and the non-significant flux increase in the (sub)mm \citep{2019ApJ...881L..39B} this promotes the G358 event to be the first NIR/MIR/(sub)mm-dark, FIR-loud MYSO accretion burst.

While the second epoch observations suffer from the lack of the blue FIFI-LS bands, which sampled the flux density at wavelengths short-ward of the SED peak, the red channel data (green symbols in Fig.\,\ref{fig:SEDS}) nevertheless unambiguously indicate elevated flux levels during the post-burst stage.


% done after first greybody call in  gbplot.pro
% fit=poly_fit(alog(bl[0:9]),alog(bf[0:9]),3)
% float(abs(bf[0:9]-exp(poly(alog(bl[0:9]),fit))))

% data source ~/Targets/G358.931-0.030/FIFI-LS/sofia_2019-05-01_FI_F562
% processed with ~/sofia_data_official_recalibration/phot.pro
% wavelengths corrected to central values
% first epoch 118µm corr. factor 1.42
% second epoch 118µm corr. factor 1.06 <- derived from 124µm ratio,scaled with 1st epoch flux, mean relative residual 11%
\begin{table}
\begin{threeparttable}
\caption[]{FIFI-LS photometry: Left first epoch, right second epoch}
         \label{fifiphot}
         \begin{tabular}{ccccc}
            \hline
            \noalign{\smallskip}
%            Central wavelength      &  Flux density & |Residual|&  Flux density & |Residual| \\
            Central      &  Flux & |Residual|&  Flux & |Residual| \\
            wavelength & density & & density & \\
            $[\mu m]$\ & [Jy] & [Jy]& [Jy] & [Jy] \\
            \noalign{\smallskip}
            \hline
            \noalign{\smallskip}
             ~~52.0 &  ~~95.6 &  ~~7.7 & &\\
             ~~54.8 & 126.8 & 15.0 & &\\
             ~~60.7 & 125.6 &  ~~5.7 & &\\
             ~~87.2 & 217.6 &  ~~4.4 & &\\
            118.6 & 255.1\tnote{*} & 37.0 & 216.5\tnote{*} & 42.6 \\
            124.2 & 371.5 & 72.8 & 304.5 & 52.4 \\
            142.2 & 315.4 &  ~~8.1 & 224.3 &  ~~2.2 \\
            153.3 & 270.4 & 34.5 & 221.4 & 12.8 \\
            162.8 & 297.1 &  ~~1.9 & 162.3 & 29.7 \\
            186.0 & 284.0 &  ~~9.4 & 155.9 &  ~~9.3 \\
            \noalign{\smallskip}
            \hline
         \end{tabular}
         \begin{tablenotes}\footnotesize
\item[*] re-calibrated value
\end{tablenotes}
 \end{threeparttable}        
\end{table}


\subsection{FIFI-LS image analysis}


% Chop angle PA, cf. Mid-IR imaging and spectroscopy with FORCAST
%CHPANGLE=              293\,000 / Calculated angle in the sky_coord_sys reference
% ABSOLUTE astrometric accuracy of SOFIA is not that great
The angular distance of 1\farcs09 between MM3 and MM1 at a position angle (PA) of 247\degr{} warrants to investigate whether both sources could be marginally resolved at the shortest FIFI-LS wavelengths, provided they have similar flux contributions. The
% 1.22*52.2\,{\times}\, 10^{-6}/2.5
diffraction limit for the 2.5-m mirror of SOFIA at 52.2\,$\mu$m amounts to 5\farcs3. However, due to various reasons \citep{2017SPIE10401E..12G}, the FWHM of the actual point spread function (PSF) exceeds that value. 

In order to address this point we cannot rely on the absolute pointing accuracy of SOFIA but have to analyze the image morphology.
%Similar to the analysis described in Sect.\,3.3, but in the opposite sense, we investigated whether the FIFI-LS data from the bands taken in the blue channel which provide the best angular resolution hint at a marginally resolved binarity, with MM3 now being the fainter component. 
%For that purpose 
So the median continuum images of the blue channel were fit by a bivariate (2D) Gaussian. If MM3 
%, situated at a position angle of ??\degr{} 1.?\arcsec{} of MM1, provides 
contributes a substantial flux contribution to the image, the major axis of the fit should clearly exceed the smaller one and be aligned to that direction. 
% Because of the wavelength dependence of the angular resolution, the axis ratio should approach unity for longer wavelengths. 
The corresponding quantities,  
axis ratio, position angle and respective errors, are given in Table \,\ref{fifibigauss}.
% based on blue_bivariate.dat, axis ratio error using http://web.mst.edu/~gbert/JAVA/uncertainty.HTML
\begin{table}
      \caption[]{FIFI-LS bivariate Gaussfit results}
         \label{fifibigauss}
         \begin{tabular}{ccc}
            \hline
            \noalign{\smallskip}
            Effective wavelength      &  Axis ratio & Major axis PA \\
            $[\mu m]$ &  & [\degr] \\
            \noalign{\smallskip}
            \hline
            \noalign{\smallskip}
% PA was wrongly calculated before
%             52.2 & $1.12\pm0.03$ &  $229.9\pm6.1$\\
%             54.9 & $1.22\pm0.02$ &  $192.6\pm1.9$\\
%             60.8 & $1.11\pm0.02$ &  $203.6\pm3.6$\\
%             87.6 & $1.10\pm0.01$ &  $209.4\pm1.1$\\
             52.0 & $1.12\pm0.03$ &  $310.1\pm6.1$\\
             54.8 & $1.22\pm0.02$ &  $347.4\pm1.9$\\
             60.7 & $1.11\pm0.02$ &  $336.4\pm3.6$\\
             87.2 & $1.10\pm0.01$ &  $330.6\pm1.1$\\
            \noalign{\smallskip}
            \hline
         \end{tabular}
\end{table}
When judging these results, it has to be kept in mind that the FIR observations were performed in chop-nod mode. Thus, deviations from a perfect image superposition may lead to an elongated image as well. %For the last three wavelengths the PA of the major axis does not deviate much from the applied chopping PA of 293\degr{} while for 52.2\,$\mu$m it is closer to that of MM3. This might be interpreted in favor of a noticeable contribution of MM3 to the flux at the shortest wavelength of the blue channel.
The PA of the major axis for the 52.2\,$\mu$m band is closer to that of MM3 compared to the other ones. However, the PAs of all other bands 
%\textcolor{red}{missing word?} 
are more aligned with the chopping angle of 293\degr{}.
%rather than MM3. 
Given this fact, it remains open whether the result for the shortest wavelength of the blue channel might be interpreted in favor of a noticeable contribution from MM3.



\begin{figure}   % Fig 2
	\includegraphics[width=\columnwidth]{G358_FIFI-LS_HI-GAL.png}
	\caption{Observed FIR/(sub)mm SEDs showing pre-burst {\em Herschel}/AKARI data (blue) and FIFI-LS burst values (red) together with the corresponding (sub)mm fluxes from \citet{2019ApJ...881L..39B}.
	%Conservative uncertainties of 10\% were adopted for fitting the FIFI-LS data, cf. \citet{2019AAS...23420805F}. 
	The upper 24\,$\mu$m MIPS limit
	%and lower 90\,$\mu$m AKARI limits are 
	is indicated as well. The solid lines represent the reddened graybody fits. The second epoch FIFI-LS data and the corresponding fit appear green.
	}
 \label{fig:SEDS}
\end{figure}

%\subsection{Spectral energy distribution}
\section{Graybody fits and burst parameters}\label{gbp}

%The  sparse SED due to the lack of flux values for the blue channel of FIFI-LS precludes to carry out the following analysis for the second epoch data.

\subsection{Graybody approximation of the FIR/(sub)mm SEDs}\label{gb}
Since the observed FIR/(sub)mm SEDs of G358 do not strongly differ from a Planck function, the most simple approach to approximate them is using a modified blackbody, i.e., graybody (e.g., \citealp{2016MNRAS.461.1328E}). %Instead of fitting  
%Both SEDs were  first fit by varying
It comprises three parameters, dust temperature, emissivity index, and solid angle of the emission.
Before performing the fits the fluxes were dereddened to account for interstellar extinction. A value of $A_{\rm V}\,{=}\,60$\,mag was assumed which seems to be justified (see Sec.\,\ref{drt1}). For 
flux dereddening 
the $R_{\rm V}\,{=}\,5.5$ dust model of \citet{2003ARA&A..41..241D} has been used.

The corresponding fit to the pre-burst SED yielded  a dust emissivity index of $\beta\,{=}\,1.84\,{\pm}\,0.06$ and a dust temperature of $T_{\rm d}^{\rm pre}{=}\,26.3\,{\pm}\,0.5$\,K which
is slightly lower 
than the pre-burst result of \citet{2019ApJ...881L..39B} 
%, and normalizing constant $c{\,=\,}\Omega \nu_0^{-\beta}$ where $\Omega$ is the solid angle and $\nu_0$ the reference frequency,
%. Since the values for $c$ and  $\beta$ were almost identical for the pre-burst and burst SEDs (0.45 vs. 0.55 and 2.19 vs. 2.18), they were set to 0.5 as well 2.2 and kept constant during a second fit.
%we adopted an opacity exponent of 1.7. It 
The emissivity index
is appropriate for a dense, relatively warm environment such as massive star forming regions for grain growth (e.g., \citealp{2005AIPC..761..123L, 2012MNRAS.422.1263H}) and temperature \citep{2005ApJ...633..272B} reasons. The dust temperature for the burst SED is marginally higher $T_{\rm d}^{\rm burst}{=}\,28.2\,{\pm}\,2.5$\,K and so is the emissivity index with $\beta\,{=}\,2.07\,{\pm}\,0.93$. Obviously, the scatter of the FIR fluxes and the lack of a pronounced peak in the burst SED
% introduce deviations which 
imply larger errors on the parameters. So within their errors, the respective quantities for both epochs seem to agree.
%This yielded temperatures of $26.4{\pm}2.1\,$K and $30.1{\pm}1.9\,$K, respectively. 
%within the errors.
% since these authors adopted a value of 1.7 for the grain opacity spectral index. 
% {\tt\red Discussion: issue for mass estimate? probably not, temp/opacity vary inversely}

An attempt was made to come up with a graybody fit for the second FIFI-LS epoch data as well. Because of the sparse data, only the solid angle was varied while mean temperatures and emissivities from the pre-burst and first epoch models were used. The fits, reddened to match the observations, are shown as solid lines in Fig.\,\ref{fig:SEDS}.


\begin{figure*}
%\centering
    \sidecaption
%	\includegraphics[width=\columnwidth]{G358_ap3_pre.png}
	\includegraphics[width=12cm]{G358_ap3_pre.png}
	\caption{Pre-burst SEDs for the following
	%radiation 
	sources: Black symbols -- total FIR/(sub)mm emission of G358, green -- 
	% SED of 
	MM3, blue -- 
	%SED of 
	MM1, derived from the total FIR/(sub)mm emission by removing contributions from all other sources (including MM3).
	%measured SED (black symbols). 
	%For all sources, except MM3, we use graybody fits to the ALMA-fluxes to estimate their respective contributions (as described in Sec. \ref{sedc}). 
	%in the FIR/(sub)mm range.
	For MM1 and MM3 the observed values are shown together with the ten best RT fits. 
	%The triangle marks a lower limit. <- removed from plot BS
	At wavelengths beyond $40\,\mu$m MM1 dominates the total flux density. The seemingly gap at $10\,\mu$m in the MM1 SED is due to strong silicate absorption.} 
 \label{fig:sed g358 pre}
\end{figure*}


\subsection{Empirical burst parameters}\label{ebp}
The FIR/(sub)mm luminosity increase due to the MM1 accretion burst can be determined by integrating the above graybody SED fits and taking source distance given in Sec.\,\ref{intro} as well as extinction into account. Here the assumption is being made that all other sources which contribute to the total G358 flux stayed constant during the pre\nobreakdash-, burst- and post-burst epochs.
%{\red
The weak NIR/MIR variability of MM3 is of no concern here since its FIR emission is far below that of MM1 (see Sec.\,\ref{fig:sed g358 pre}).
%}
% Given that MM1 is deeply embedded, an 
Since the bulk of the energy is emitted in the FIR, the FIR/(sub)mm luminosity estimates only weakly depend on $ A_{\rm V}$. 
%The distance uncertainty will be propagated in the following analysis.
The impact of the distance ambiguity is considered in the following analysis.

% latest results from gbplot
Integration of the graybody fits yields the following FIR/(sub)mm luminosities: 
%(in units of $10^3\,{\rm L}_\odot$):
Pre-burst $L^{\rm pre}_{\rm FIR}\,{=}\,7600\,{\mypm}\,^{~800}_{1400}\,{\rm L}_\odot$, first FIFI-LS epoch $L^{\rm burst}_{\rm FIR}\,{=}\,19\,300\,{\mypm}\,^{2200}_{3700}\,{\rm L}_\odot$, and second epoch $L^{\rm post}_{\rm FIR}\,{=}\,12\,700\,{\mypm}\,^{1500}_{2600}\,{\rm L}_\odot$, respectively. 
%In the absence of time-resolved FIR photometry we have to take the latter as representative for the whole burst duration. 
As emphasized, without interstellar extinction, these values shrink but only by 9\%.
%change slightly to $7.0\,{\mypm}\,^{0.8}_{1.3}$, $17.5\,{\mypm}\,^{2.0}_{3.4}$, and $L^{\rm post}_{\rm FIR}\,{=}\,11.7\,{\mypm}\,^{1.3}_{2.2}$. 
The luminosity increase due to the accretion burst at the dates of the FIFI-LS observations amounts to $\Delta L^{{\rm burst}}_{\rm FIR}\,{=}\,11\,700\,\mypm^{\,2300}_{\,3900}\,{\rm L}_\odot$ and $\Delta L^{{\rm post}}_{\rm FIR}\,{=}\,5100\,{\mypm}\,^{\,1700}_{\,2900}\,{\rm L}_\odot$. The presence of a substantial luminosity increase during the second FIFI-LS epoch, i.e., about 18 months after the peak of the maser flare, is remarkable.

% use result from linear equation 862d or median 822d here?
% median
For deriving major parameters of the burst we follow the approach of \citet{2017NatPh..13..276C}. In addition, for what concerns the estimate of the burst energy, the two epochs of FIFI-LS observations provide the opportunity to account for the temporal evolution of the luminosity. The simplest, yet plausible, approach is to assume a linear decrease which holds from the flare peak date to the date when the pre-burst level will be reached again. The linear flux decay approximation yields a duration $\Delta t$ of $869\,{\pm}\,303$ days, with a pre-burst-level return date of 2021 July 31.

The FIR/(sub)mm burst energy is $E^{\rm acc}_{\rm FIR}\,{=}\,{<}\Delta L^{\rm acc}_{\rm FIR}{>}\,{\times}\,\Delta t$, where ${<}\Delta L^{\rm acc}_{\rm FIR}{>}$ is the average luminosity increase which equals to half of the peak increase for a linear drop to zero.  It amounts to $E^{\rm acc}_{\rm FIR}\,{=}\,1.9\,{\times}\,10^{38}$\,J. Its upper and lower bounds from the uncertainties in both duration and luminosity are $1.0\,{\times}\,10^{38}$\,J and $2.6\,{\times}\,10^{38}$\,J, respectively. 
%Notably, this estimate for the burst energy is about twice as much as the peak luminosity would yield over the duration of the maser flare which lasted for about four and a half months (G. MacLeod et al., in prep.).




%The luminosity increase due to enhanced accretion $\Delta L^{\rm acc}_{\rm FIR}\,{=}\,L^{\rm burst}_{\rm FIR}-L^{\rm pre}_{\rm FIR}$ amounts to $\Delta L^{\rm acc}_{\rm FIR}\,{=}\,(1.19\,\mypm^{\,0.23}_{\,0.39})\,{\times}\,10^4\,{\rm L}_\odot$.
%and assume that the luminosity increase is solely due to activity of MM1, i.e., all other sources of the G358 complex stayed constant. 
%The burst energy $E^{\rm acc}_{\rm FIR}\,{=}\,\Delta L^{\rm acc}_{\rm FIR}\,{\times}\,\Delta t$, where $\Delta t$ is the length of the burst, can be estimated using
%%, obtained from the pre- and outburst SEDs, 
%a burst duration of $\approx$3 months as evidenced by the maser activity (MacLeod et al., in prep.). It amounts to $E^{\rm acc}_{\rm FIR}\,{=}\,(3.6\,{\mypm}^{\,0.69}_{\,1.12})\,{\times}\,10^{37}$\,J.

We emphasize that these values represent lower limits to the  luminosity increase and the corresponding energy release since they are based on the FIR/(sub)mm emission only. The ultimate quantities will be derived from the results of the RT analysis below.
Accretion related quantities require the knowledge of protostellar mass and radius and will be considered in Sec.\,\ref{sbp}.

\section{Analysis of spectral energy distributions}\label{ased}

%\subsection{On Heating and cooling of the circumstellar environment}
%In the steady state, i.e., at constant source luminosity, both circumstellar disk and envelope attained certain temperature distributions, established by RT through absorption and re-emission. 




\subsection{SED decomposition}\label{sedc}
%\textcolor{red}{For the empirical derivation of the luminosity increase due to the burst integrated fluxes for the G358 region have been used. This is a valid approach given the evidence that it was caused by MM1 alone, i.e., all other sources varied negligibly or not at all. 
%However, i}\textcolor{green}{evtl. Satz streichen}
%\textcolor{green}{the MM1-SED has been reconstructed by subtracting the contribution from the other sources from the total values.}
%Only for MM3 (which is rather evolved and luminouse in the NIR/VIS) it was possible to use RT-modeling to obtaine the SED. We use the ALMA-detections to fit the other sources with greybodys. Since the other sources lack the detection in the NIR, we assume, that they are relatively young, which justifies the greybody simplification. 
%The reconstructed MM1-fluxes amount to approximately half of the total pre-burst-flux.}%

In order to derive representative parameters of the bursting MYSO MM1 from RT modeling, the underlying SED should be as free as possible from contributions from other objects. Thanks to the availability of NIR/MIR and (sub)mm photometry for MM3, the contamination from this source to the overall flux can be removed by using predicted flux densities from its best RT model (see Sec.\,\ref{drt3}), assuming that its SED has not changed. 
%for this object will be established first, and then subtracted from each joint SED.
This approach has been proven successful in a similar, yet less sophisticated fashion, for a study of FIR emission from W3(OH) and the neighboring W3(H$_2$O) (\citealp{2002A&A...392.1025S}). 

Here we extend it by also taking into account the presumed contributions from the remaining sources detected by ALMA/SMA. Since none of these has an IR counterpart we assume that they are in an early evolutionary stage similar to MM1, with SEDs that resemble the overall pre-burst FIR/(sub)mm SED. So for MM1 and each of those (MM2, MM4-8) the ALMA/SMA (sub)mm fluxes from \citet{2019ApJ...881L..39B} were fitted by a graybody, using the temperature and emissivity index derived from the pre-burst SED, to obtain the individual solid angles of the emission. For the pre-burst epoch, the following steps were performed to obtain the MM1 fluxes. First, the predicted MM3 fluxes were subtracted from the G358 total fluxes. Second, the ratio between the MM1 solid angle and the sum of all solid angles was calculated which amounts to $0.50\,{\pm}\,0.05$. It corresponds to the relative contribution of MM1 to the MM3-subtracted fluxes. So, by multiplying the latter with this ratio the pre-burst fluxes of MM1 were obtained.

Similarly, for the wavelengths of the burst as well as post-burst observations the pre-burst fluxes were predicted from the MM3 model as well as the pre-burst graybody fit. Then, the MM3 contribution was removed. Finally, the contribution of MM2+MM4-8 to the MM3-subtracted pre-burst fluxes had to be taken out from the observed burst as well as post-burst fluxes to yield those for MM1.
%Similarly, the MM3 contribution predicted by its best model as well as the total flux from the remaining sources, obtained by taking their total solid angle into account, were subtracted from the measured fluxes to yield those for MM1.
These are listed in Table \ref{tab: preSED}, \ref{tab: burstSED} and \ref{tab: postSED} (for MM1 pre-, burst and post-burst respectively).
% We note that 
%The ratio of the MM1 solid angle to the a correction factor had to be derived beforehand and
%}
%applied to reproduce the total flux. It is of similar value as the ratio of the integrated flux density of G358 at 889\,$\mu$m of 1.13\,Jy to the sum of the fluxes from all individual components of 0.61\,Jy. 
They
%so refined pre-, burst and post-burst SEDs 
represent the best possible approximation of the intrinsic ones of MM1.
We emphasize that the procedure to derive the MM1 fluxes is tailored to reproduce the total flux. Therefore, the MM1 (sub)mm fluxes exceed those given in Tab.\,2 of \cite{2019ApJ...881L..39B}.



\subsection{Radiative transfer analysis}\label{rta}
Characteristic properties of YSOs, namely their luminosities as well as mass, geometry and extent of the surrounding dust, can be derived by modeling their dust continuum radiation to match the observed SEDs (e.g., \citealp{2012ascl.soft04005W, 2013ApJS..207...30W}). In the case of G358 this has been done by \citet{2019ApJ...881L..39B} to infer the  pre-burst luminosity
%range of 
$L^{\rm pre}$,
%{\approx}5700\dots22000\,{\rm L}_\odot$ 
using the YSO model grid of \citet{2017A&A...600A..11R}. 

Utilizing the same model pool we performed an RT analysis of the SEDs of MM1 and MM3 using the Python implementation `sedfitter'\footnote{\url{https://zenodo.org/record/235786}} of the SED fitter \citep{2007ApJS..169..328R} which is described below. We did not fit the G358 total flux (black symbols in Fig. \ref{fig:sed g358 pre}), since %neither the 'spubhmi' nor the 'spubsmi'\footnote{\textcolor{green}{Both settings are composed of the following components: star, circumstellar disk, ulrich-type envelope and ambient medium, where all parameters are sampled from typical ranges. The 'spubhmi'-setting features an inner hole in between star and disk, whereas for 'spubsmi' the inner radius is governed by the dust sublimation radius (assuming $T_{sub}=1600\,{\rm K}$). For further details we refer the reader to \citet{2017A&A...600A..11R}}} setting includes 
models with multiple sources, which would be appropriate for massive star forming regions and clearly for G358, are not included in the \citet{2017A&A...600A..11R} model grid.
Instead, we extracted the fluxes of MM1 from the total ones as described above.
%As mentioned before, the flux of MM1 is contaminated by MM3. 
%\textcolor{red}{
From the 10 best fits (per epoch/source) mean values and uncertainties for all 'free' parameters of the models 
%in the grid
are derived (see Appendix \ref{app}).
%}
%All mean parameter values, which we derive from this analysis, are based on the 10 best fits.
Since all members of the model pool have 9 different inclinations, the 10 best fits are not necessarily composed by 10 different models. Instead, some models might be included more than once, but with different inclinations.
By using a weighting with the respective ${\goodchi}^2$ values, we ensure that the best fits determine the corresponding mean values. For the parameters which are log-spaced in the Robitaille model pool ($r_{inner},\, r_{outer},\, M_{\rm disk}^{\rm dust},\,  L$\, via $R_\star,\, T_\star$), we use the geometric mean, while for all other parameters, we use the arithmetic mean.

%\textcolor{green}{
We note that these models incorporate passive disks. While this may seem inappropriate in the context of accretion bursts where active disks are often invoked, it is justified by the fact that we are primarily interested in reproducing the FIR/(sub)mm emission. Due to the strong radial dependence of their viscosity
%in active disks
, e.g.,  \citet{1981ARA&A..19..137P}, active disks differ from passive ones primarily in the very innermost region where the bulk of the dissipated energy is being released. The details of this process are not relevant for the highly reprocessed FIR/(sub)mm emission, 
%which arises from dust in
which is dominated by dust radiation from the outer regions.
%}

Before describing the actual modeling we emphasize that
additional data and results are provided in the Appendix \ref{app}. These comprise the flux tables which were used to establish the SEDs, a summary of the RT models which have been used in the analysis, tables listing the parameters of the ten best RT models for each of considered cases, and a table providing optical depths for the ten best pre-burst RT models of MM1.


%\textcolor{red}{The SED of MM3 is well constrained by its NIR/MIR photometry together with the ALMA flux densities. Therefore, it is possible to apply RT modeling to MM3. This will not only allow us to draw conclusions on MM3 itself, but it is also required to disentangle the fluxes of MM3 and MM1.} \textcolor{green}{delete this? It is written above!}


\subsubsection{Modeling of the dust continuum radiation of MM3}\label{drt3}
 
For the fitting of the MM3 SED the combined `spubhmi' + `spubsmi' data sets from \citet{2017A&A...600A..11R} have been used. They comprise 120\,000 YSO models with 9 inclinations for each model 
%\textcolor{green}{
(leading to a set of 1\,080\,000 SEDs in total).
%}
The designations are based on the respective model components.
Each model (in both data sets) comprises the following components: {\bf s}tar, 
%{\red
{\bf p}assive circumstellar disk,
%}
{\bf b}ipolar cavity, {\bf U}lrich-type envelope \citep{1976ApJ...210..377U}, and ambient {\bf m}edium. % according to the structure of a YSO at a very early evolutionary stage. 
The 'spubhmi'-setting features an inner {\bf h}ole in between star and disk, whereas for 'spubsmi' the inner radius is governed by the dust {\bf s}ublimation radius (assuming $T_{\rm sub}\,{=}\,1600\,{\rm K}$). Note that the Robitaille model pool includes other data sets, that are composed of less (or different) components and are thus less suited to represent the structure of a YSO at a very early evolutionary stage. For further details we refer the reader to \citet{2017A&A...600A..11R}. 
%{\red
A synthetic aperture size of $3\arcsec{}$ has been used for fitting the SED. The fluxes of the MM3 SED are listed in Tab.\,\ref{tab: MM3 SED}. 
%}

We adopt the same distance range as for MM1 but allow for a smaller extinction.
Figure \ref{fig:sed g358 pre} shows the SED (green) together with the total FIR/(sub)mm pre-burst SED (black symbols) and the pre-burst flux attributed to MM1 (blue), where the contribution of the other sources has been removed as described in Sec. \ref{sedc}. The ten best fits for each SED are shown with solid lines. The best fit is dark, whereas the other are slightly transparent.
Our models suggest that the contribution of MM3 to the total flux at wavelengths beyond $\lambda\,{\ge}\,40\,\mu$m is marginal. Nevertheless we take the best MM3 fit into consideration when refining the MM1 fluxes in the following analysis. 
%Note that the best-fit models shown in blue refer to the refined 'MM1'-fluxes, whereas the data-points indicate the measured values, which include all sources.
%}

As indicated by its moderate obscuration in the NIR
%/VIS 
already: MM3 is likely the most evolved object of the G358 complex. 
It features a 
%relatively heavy
disk with a dust mass of $M_{\rm disk}^{\rm dust}\,{=}\,0.068\,{\mypm}\,_{0.043}^{0.11} \,{\rm M}_\odot$. This may seem heavy keeping the canonical gas-to-dust mass ratio ($\gamma$) of 100 in mind. However, the latter is not applicable since $\gamma$ depends on the galactocentric distance $R_{\rm GC}$ \citep[Eq. 2,]{2017A&A...606L..12G} because of the Galactic metallicity gradient. For G358 $R_{\rm GC}{=}$1.6\,kpc implies a value of $\gamma\,{=}\,38\,{\pm}\,4$ which yields a total disk mass of $2.6\,{\mypm}\,_{1.0}^{6.8}\,{\rm M}_\odot$.
The fit delivers $A_V\,{=}\,20\pm 5\,$mag and an inclination of $i\,{=}\,51\,{\pm}\,9$\degr. The inner radius of the disk seems to be close to the star, 
%\textcolor{red}
%{
$r_{\rm inner}\,{<}\,3.5\,R_{\rm sub}$ holds for each of the 10 best models
%}
. The mean value of the disk outer radius amounts to $510\,{\mypm}\,_{240}^{450}\,$au, while the maximum is as high as  $1600\,$au.
The mean luminosity of $7540\,{\mypm}\,_{3130}^{5340}\,{\rm L}_\odot$ corresponds to a ZAMS star of $11\,{\rm M}_\odot$ and $4.2\,{\rm R}_\odot$ \citep{1996MNRAS.281..257T}. While most of the models have luminosities below $10\,000\,{\rm L}_\odot$, one has a luminosity of $43\,000\,{\rm L}_\odot$. Such a high effective temperature would imply the presence of a compact H{\sc ii} region. However, MM3 escaped the detection in the radio continuum at a sensitivity level of ${\approx}\,50\,\mu$Jy\,beam$^{-1}$
%{\red
%(synthesized beam size $3\arcsec\,{\times}\,6\arcsec$)
in the survey of
%}
\citet{2016ApJ...833...18H} which probably rules out this model. Recently, Bayandina et al. (in prep.) succeeded to detect faint radio continuum emission from MM3.

%We will report about selected parameters in the following. 
The whole parameter set for the 10 best models (including their respective $\goodchi^2$ values) can be found (together with the weighted means and standard deviation $\sigma$) in Table \ref{tab MM3 fit}.

\subsubsection{Modeling of the dust continuum radiation of MM1}\label{drt1}

\begin{figure*}   % Fig 2
%	\centering
    \sidecaption
%	\includegraphics[width=\columnwidth]{G358_ap3_mean.png}
	\includegraphics[width=12cm]{G358_ap3_mean.png}
	\caption{Modeled pre-burst MM1 SEDs (blue), together with its burst (red) and post-burst SED (green). Triangles mark upper limits. The mean-model (see text) is shown in black. It is shown three times but with different source luminosities for the respective observing epochs. 
	%The upper limit at $2.15\,\mu$m is outside the plot range. 
	Since the $870$ (pre-) and $889\,\mu$m (post- and burst) fluxes are very similar, their symbols cannot easily be distinguished.%The sub-mm burst fluxes (red crosses) have been ignored for the SED fit (see text).
	}
 \label{fig:sed g358}
\end{figure*}

With the refined MM1 SEDs at hand, a comparison of the results from RT modeling for the pre-, post- and burst-states becomes possible.

Before describing this analysis, a remark must be made concerning the interstellar extinction $A_V$ which is a free parameter of the SED fitter. Because of the wavelength dependence of the dust optical properties, extinction is most pronounced at short wavelengths. Thus, for SEDs like that of MM3 it can be well constrained by the best models. However, for the SED of MM1 which is almost exclusively defined by FIR and (sub)mm measurements, interstellar extinction is less influential and, therefore, harder to derive. Since higher $A_V$ implies larger source luminosities to reproduce the observed fluxes, an upper limit needs to be established to constrain $L$. The re-calibration of a Galactic dust-reddening model based on IRAS and COBE/DIRBE results by \citet{2011ApJ...737..103S} suggests a value of $A_V\,{=}\,115\,{\pm}\,4.2$\,mag along the sight line toward G358. Since this holds for the {\em whole} path across the Galaxy, while G358 is in front of the Galactic center region, the actual $A_V$ will in fact be lower. Therefore, we assumed an extinction range of $A_V\,{=}\,30\,{-}\,70$\,mag.


To begin with, we fitted the pre-burst MM1 SED which represents the stationary state using the distance and extinction ranges given above and established so the 10 best pre-burst 
%\textcolor{green}{
fits (i.e., models with fixed inclination).
%} %models. % from that.
%assuming a distance from $d=6.75{\mypm}_{0.68}^{0.37}\,kpc$ \citep{2019ApJ...881L..39B}
%and assuming an extinction range of $A_V=30\dots70$\,mag. The high extinction is indicated by the non-detection of MM1 at NIR/MIR wavelengths. 
For this purpose 
%the Python implementation of the SED fitter \citep{2007ApJS..169..328R} has been used in connection with 
the 'spubsmi' data set from \citet{2017A&A...600A..11R} has been used which includes 
40\,000 models 
%\textcolor{green}{
at 9 inclinations (360\,000 SEDs in total).
%}
For these models the inner radius of the dust disk corresponds to the sublimation radius $R_{\rm sub}$. This constraint seems to be justified given the high accretion rates at which massive stars form (e.g., \citealp{2007ARA&A..45..481Z, 2016ApJ...832...40K}), and in particular before and during an accretion burst. It requires a few remarks concerning the understanding of `heating' and `cooling' in this case, since the dust disk cannot get any hotter than the sublimation temperature. What happens due to a luminosity increase is that enhanced dust sublimation at the inner rim $R_{\rm sub}$ pushes the latter outward. This is accompanied by an adjustment of the temperature profile $T(r)$ via absorption and re-emission such that, for a given radius (beyond the actual $R_{\rm sub}$), the temperature exceeds the previous value. Conversely, for a given temperature, the growth in radius implies a larger radiating area and, therefore, leads to a flux increase. 
The reverse process happens once the burst ceased. $R_{\rm sub}$ will shift back inwards, allowing the dust replenishment by accretion and/or dust re-formation. This will lead to a flux decrease and might be considered as `cooling' but, yet, the inner rim is still at the dust sublimation temperature.
%\textcolor{blue}{Das ist natürlich ein bisschen eine Frage der Definition; Am festen Radius r>Rsub gibt es schon einen Temperaturanstieg/abfall (so, wie man sich das vorstellt, wenn man an heizen denkt), wenn man sich die Gebiete konstanter Temperatur anschaut, hast Du andererseits recht. Ich finde den Absatz gut und sehr illustrativ, würde aber die Passage nach dem Burst leicht abändern. Natürlich kann der Staub nur 'zurück' kommen, weil die Temperatur wieder gefallen ist, bzw. Rsub wieder nach inne wandert.}

%We use the sedfitter with the 'spubhmi'-setting introduced in \citet{Robitaille2017} to fit the pre-burst-sed. The 'spubhmi'-dataset includes 72\,000 SEDs in total with all of the following components: star, disk, cavity, ulrich envelope and ambient medium. Each values are sampled from typical ranges. For more details, we refer the reader to \citet{Robitaille2017}. We assumed a distance from $d=6.75 \mypm_{0.68}^{0.37} kpc$ \citep{Brogan2019} and a high extinction $A_V=50..70 mag$.
The next step was to establish a new model pool from the 10 best pre-burst-SEDs which serves to fit the post-/burst SEDs. This pool contains 100 SEDs in total, where we re-use the best 10 models, with a source luminosity increasing in 9 linearly spaced steps from $2$ to $6\,L^{\rm pre}$, respectively. %\textcolor{green}{
Inclination, interstellar extinction and the distance
% of the models 
were set to the values of the underlying model from the pre-burst-fit
%result. This was done, 
since none of these parameters is expected to change during the burst.
%}
Additionally, the original pre-burst-SEDs are included. 
%A schematic visualization can be found in Fig. \ref{fig: setting}. Due to the burst the source luminosity increases. We assume, that the increase is due to stellar bloating (and not due to heating). The difference between bloated and heated source occurs mainly in the VIS/NIR, where only upper limits exists, which lay well above all curves. Thus this assumption will not change the results. 
%Furthermore we assume that all 
Since the inner disk radius is governed by the dust sublimation temperature, assumed to be $1600\,$K, it shifts outward for those models with increasing luminosity. Otherwise, the system geometry is kept the same.
While this ignores possible changes of the disk due to the burst, e.g., in the density structure, it is the simplest approach to model the dust continuum emission due to the accretion burst and feasible to be treated using common RT codes which are generally static. 
While a proper burst modeling requires time dependent RT, possibly coupled to hydrodynamics for utmost consistency, our simplified treatment nevertheless allows us to draw major conclusions.

For computing the burst models, the Hyperion code 
\citep{Robitaille2011} has been used. 
The so obtained database is the foundation to fit the SEDs of burst and post-burst epochs. 
We note that by constraining the model pool we ensure consistency of the results, i.e., the best fits for all - pre-burst, burst and post-burst SEDs - are based on the same models. 
%\textcolor{red}{
A sketch of all SEDs that are included in burst model database, can be found in the Appendix (Fig. \ref{fig:sed g358 modelpool}.) The range of the observed fluxes from the burst- and post-burst-epoch is well covered by the models. Only the burst observations at $\lambda\,{=}\,163$ and $186\,\mu$m are outside of the flux range covered by the models.
%} 
%However all model SEDs show a shallower slope in between 160 and $900\mu m$ than the observations. Therefore the best fits will very likely underestimate the flux at those wavelengths, while the sub-mm flux is overestimated. 
%However {\tt\red Warum however hier?}\\
Only one model out of the 10 best burst fits has the maximal luminosity increase included in the pool. Therefore it is not necessary to include models with higher source luminosities.

Before presenting the results, a few remarks have to be made concerning the fitting. We exclude the sub-mm observations at $\lambda>890\mu m$ from the SED-fit of the burst since their deviation from the stationary models is biggest at those wavelengths (see Sec. \ref{misf}). The post-burst is observed only at wavelengths greater than $118\,\mu$m, meaning that there are no constraints in the MIR. Since the maser flux did not fall below its pre-burst level until now (MacLeod et al., in prep.; Yonekura et al., in prep.), we assume that the post-burst flux in the MIR is also not below the pre-burst level. Therefore, models from the post-burst fits that have MIR fluxes below those of the mean pre-burst model were excluded.
% {\tt\red habe sie weggelassen da die Erläuterungen zur Modellierung eh schon komplex sind}
An aperture size of $3\arcsec{}$ has been applied in for the fits. With this choice the $\goodchi^2$-value of the pre-burst fit becomes smallest. All obtained parameters are stable against a variation of the aperture size (they agree within their respective errors).


%{\tt\red folgendes verstehe ich noch nicht ganz}
%\textcolor{red}{Note that for the mean pre-model (used to exclude models with too low MIR fluxes only) we obtain the mean parameters from the pre-burst alone, where we include the burst and post-burst-fit in our results (given in the text or Table \ref{tab MM1 fit}).}
%\textcolor{green}{Es gibt einen Unterschied zwischen Mean in fig 6 und mean pre in Fig A.1.. mean pre basiert nur auf den 10 pre-modellen, während Mean auf allen 30 Modellen (pre, burst, post) basiert. mean pre wurde ausschließlich dazu verwendet, um ein unteres Limit für den MIR post-burst Fluss festzulegen. Evtl. können wir diese Bemerkung auch ganz rauslassen.}
 %, contrary to previous works where pre- and burst YSO models were drawn from the full data set. 
 %Furthermore, ... comment on aperture sizes?

 
%In that code it is not possible to include time-dependent heating/cooling of the disk. To fully understand the processes within the disk time-dependent modeling is necessary. Nevertheless, it is possible to obtain some first conclusions from this simple procedure.

The results of the RT modeling of the MM1 SEDs are visualized in
Fig.\,\ref{fig:sed g358}, which shows the 10 best fits to the pre- (blue), burst (red) and post-burst SED (green).
The best model is shown with the darkest line, while the nine other ones are slightly transparent.
In addition to the 10 best fits, the so called 'mean'-model is shown in black. This model is computed using the weighted mean parameters from the combination of pre-, post- and burst-fits (see below). The stellar parameters, which are obviously changing during the burst, are defined by the pre-burst models alone. The mean model is not only shown for the pre-burst, but also for source luminosities
%increased by the 
corresponding to the
mean values during post- and burst. As expected, the mean model lies within the range, spanned by the 10 best models at each epoch. The results are summarized in Table \ref{tab MM1 fit}.
%In the sub-mm the measured burst-values (indicated with the red crosses) are below the best fit pre-burst SED. Since our new data-set only includes the adapted pre-burst models (with increased source luminosity's), all models will overestimate the sub-mm fluxes. For the burst-fit we therefore ignore the sub-mm fluxes. With this we got a good fit to the MIR-values. 


% An luminosity increase by a factor of \textcolor{red}{$4.7\,{\mypm}\, _{1.5}^{2.1}$} is indicated, from % $L^{\rm pre}{=}4220\,{\rm L}_\odot$ to $L^{\rm burst}{=}14600 \pm 4700\,{\rm L}_\odot$.

%In the following luminosities are given again in units of $10^3\,{\rm L}_\odot$.
%{\tt\red das mit der weiteren Stelle war nicht ganz korrekt, ich runde mal auf volle 100}\textcolor{green}{ok}\\
The RT modeling indicates that during the burst the luminosity increases from pre-burst level of $L^{\rm pre}\,{=}\,5000\,{\mypm}\, _{\,900}^{1100} \,{\rm L}_\odot$ to $L^{\rm burst}\,{=}\,23\,400\,{\mypm}\,_{3700}^{4400}\,{\rm L}_\odot$.
%in the first epoch. 
At the post-burst epoch, the luminosity is still elevated at a level of $L^{\rm post}\,{=}\,12\,400\,{\mypm}\,_{1700}^{2000}\,{\rm L}_\odot$.

% The SED fits indicate an intermediate disk inclination in the range of $40\dots60\deg$. The median disk mass of the results is $1.4E{-}6\,{\rm M}_\odot$ {\tt\red problematisch, {\bf geringer} als die akkretierte Masse!}, the median maximal extent is $1400\,au$.
%\textcolor{red}{
This corresponds to a relative luminosity gain by a factor of 
%\\ {\tt\red
%\frac{L^{\rm burst}}{L^{\rm pre}}\,{=}\,
% relativ steht ja da, sollte man verstehen ;-)
%} \\
$4.7\,{\mypm}\, _{1.5}^{2.1}$ during the burst and by
%a factor of  
%$\frac{L^{post}}{L^{pre}}\,=\,
$2.5\,{\mypm}\, _{0.8}^{1.1}$ for the post-burst epoch.
% {\tt\red das ist eine Frage was man unter relative gain versteht, L\_burst/L\_pre-burst oder (L\_burst - L\_pre-burst)/L\_pre-burst ?}\\ 
The increase in luminosity during the burst amounts to  $\Delta L^{\rm burst}\,{=}\,18\,000\,{\mypm}\,_{4900}^{6100}\,{\rm L}_\odot$ which implies a total burst energy of 
%\textcolor{red}{
$E_{\rm acc}\,{=}\,2.9\mypm_{0.8}^{1.1}\,{\times}\,10^{38}$\,J, where the calculation has been done similar to Sec. \ref{ebp}. 
%The linear extrapolation of the luminosity to the onset {\tt\red onset oder flare peak date? müssen uns einigen}\\
%of the maser flare leads to $L^{\rm burst}_{\rm max}\,{\rm}\,{=}\,19\,000\,L_\odot$ with a 
The derived linear decay time of $\Delta t\,{=}\,907\,{\mypm}\,_{236}^{318}$\,d predicts an expected return to the pre-burst luminosity in 2021 September, which agrees with the previous estimate based on the gray-body fits within the errors (see Sec.\,\ref{ebp}).
%} 
%}
%\textcolor{blue}{

A comparison of the RT results to those obtained from graybody fits 
%to the FIR/(sub)mm emission (considered to be a lower limit) 
in Sec. \ref{gb} shows that the luminosity increase and energy release from the RT models are indeed higher, although the luminosities at burst and post-burst epochs agree within their respective errors. 
% We emphasize that t
The main reasons
%for a {\em triple} value of the accretion burst energy from the RT analysis over the graybody approach 
are the inclusion of more energetic emission from hotter regions in the models which lacks in the graybody fits and the correction of the flux contribution from the other G358 sources as outlined in Sec.\,\ref{sedc}.
%Note that the decay-times are very similar.
%}
%$E_{\rm acc}\,{=}\,(1.0\,{\mypm}_{0.5}^{1.3})\,{\times}\,10^{37}$\,J. \textcolor{red}{With $\Delta t=840 \mypm _{170}^{520}$ days (assuming a linear decrease of the luminosity until the pre-burst-level is reached).}}

The estimated parameters from the RT modeling are given below.
The foreground extinction indicated by the fits is $A_V\,{=}\,60\,{\pm}\, 10$\,mag. The system has a low inclination of $22\,{\pm}\,11\degr$, i.e. is seen close to pole on. This viewing geometry agrees with that of the spiral-arm accretion flow of \citet{2020NatAs.tmp..144C} with $i\,{=}\,25\,{\pm}\,10$\degr{}.
%{\red as well as a kinematic model ... LEAVE OUT IF NO REPLY FROM ROSS!}
The stellar radius amounts to $8.4\,{\mypm}\,_{5.5}^{15.7}{\, \rm R}_\odot$. It has been obtained from the pre-burst-models alone. The burst might cause an increase in the stellar radius (bloating) although it is unclear, whether the protostar will respond on time scales of a few months.
%Although
The models show a considerable scatter in the derived disk properties. The derived outer radii are between $140$ and 3800\,au, with a mean value of $950\,{\mypm}\,_{580}^{1500}$\,au. 
% {\tt\red bigger than MM3! but within the errors}\\
The favored dust mass of the circumstellar disk is as low as $8.4\,{\times}\, 10^{-5}\,{\rm M}_\odot$, where the $1\sigma$-confidence interval extends from $1.1\,{\times}\, 10^{-6}$ to $6.1\,{\times}\, 10^{-3}\,{\rm M}_\odot$. 
%\\{\tt\red the mean is only the most probable value if the probability distribution is symmetric!}
%{\red
Using the appropriate gas-to-dust mass ratio (see Sec.\,\ref{drt3}) this corresponds to a most probable total disk mass of 
$3.2\,{\times}\, 10^{-3}\,{\rm M}_\odot$ within an uncertainty range of $4.4\,{\times}\, 10^{-5}$ to $0.24\,{\rm M}_\odot$.
%All models, except one, have dust disk masses on the order of $1.0\,{\times}\, 10^{-3}\,{\rm M}_\odot$ or smaller. %Only one model indicates a mass of $0.036\,{\rm M}_\odot$, which is still one order of magnitude lower than the maximum disk mass of the model pool. %\sout{One model has a disk mass of $4\,{\times}\, 10^{-8} {\rm M}_\odot$, which is among the lowest disk masses in the pool.}
Remarkably, the corresponding total disk masses are lower, and mostly much smaller, than those derived for MYSOs from SED fitting (e.g., \citealp{2010Natur.466..339K, 2015ApJ...813L..19J}).  
 %\st{and by far less than disk masses derived for HMYOs (\cite{2010Natur.466..339K}, \cite{2015ApJ...813L..19J}).}
Among the best models there is only one (90Yt0exl\_03) with a total disk mass of $1.4\,{\rm M}_\odot$ which is not that much below present estimates. In contrast to MM3, the much higher scatter of the disk mass of MM1 compared to the other log-spaced parameters indicates that it cannot be reliably estimated by the fitter. In other words, a substantial flux contribution from a disk is not necessarily required to reproduce the SEDs. Thus, while we cannot reliably estimate its mass, we may conclude that the disk is quite likely lightweight.
%}

Note that during the burst the disk dust mass might decrease due to dust sublimation. The total mass will be unaffected, assuming that all sublimated dust just increases the gas mass. %The burst disk dust masses in Table \ref{tab MM1 fit} are the dust masses from the disk in the underlying pre-burst-models. Since we use a sublimation temperature of $T_{\rm sub}\,{=}\,1600$\,K the 'real' disk dust masses contributing to the burst model SEDs are somewhat lower. The same is in principal valid for the envelope (where $r_{\rm min}$ shifts to $R_{\rm sub}^{\rm burst}$ as well), although we don't give masses, but only densities.
The whole parameter set for the 10 best models for each of the three epochs (including their respective $\goodchi^2$ values) can be found (together with the weighted means and standard deviation $\sigma$) in Table \ref{tab MM1 fit}.

%\\ {\tt\red In der Tabelle sind nur die Scheibenmassen aber nicht Scheibe+Hülle!} \\
The above given properties have been derived from the results of all MM1 SED-fits, pre-, post- and burst. The respective $\goodchi^2$-values were used to weight the obtained parameters, similar to what has been done for MM3. We normalized the sum of the respective weighting factors to unity, 
%\textcolor{red}{
split up to 0.5 for the pre-burst and 0.25 for the other two epochs.
%{\tt\red here comes 'stationary' which has not been mentioned before} \textcolor{green}{das stimmt, sollen wir stationary vorher erwähnen oder hier eher den 'vollen Modelpool' mit 40.000 modellen und 'burstdatenbank' mit 100modellen erwähnen?}\\
With this we ensure an equal contribution of the fit to the 'stationary'- (pre) and 'non-stationary'-system (post- and burst), where the 'non stationary'-contribution is obtained taking into account burst and post-burst epoch equally.
%}
%With this choice of the weighting factors we account for the fact that the $\goodchi^2$-values of post-/ and burst fits are about one order of magnitude above the ones of the pre-burst fits. While we ensure, that the best models reproduce all cases (pre-, burst and post-burst).
%However even if
%Although the burst fit is not as good as the pre-burst fit, the best models should reproduce both cases.
The $\goodchi^2$-values of the burst models are about 5 times higher than for the pre-burst. This might be related to the scatter in the FIFI-LS data, to a coarser sampling of the model-parameters (because of the fact that we only consider models that fit the pre-burst SED pretty well), and to differences in comparison to the static case (whereas the pre-burst-system most probably is stationary, this does not hold for the post-/bursting stage).
Note, that in the burst case the relative weights of the models are on the same order, whereas in the pre-burst case they differ by a factor of 1.5 at most (for the post-burst it is 1.3).

\section{Stellar-dependent burst parameters}\label{sbp}
%{\tt\red The stellar-dependent part has to be revised/shifted once the "decomposed" MM1 luminosity is known}
The RT modeling of the MM1 SEDs aimed at determining the luminosities of the MYSO for the various epochs to infer the total energy released by the accretion burst.
%, in particular its pre- and burst  luminosity. 
For deriving the accreted mass and the mass accretion rate, protostellar mass and radius have to be known. For YSOs approaching the ZAMS the corresponding values which match the  luminosity are a good approximation. At first we use this approach to obtain mass and radius estimates
% solzams.pro
which reproduce the MM1 pre-burst luminosity, assuming solar metallicity. For
%$L^{\rm pre}{\approx}5.43{\times}10^3\,{\rm L}_\odot$ 
%$L^{\rm pre}\,{\approx}\,5.0\,{\times}\,10^3\,{\rm L}_\odot$
$L^{\rm pre}\,{=}\,5000\,{\mypm}\, _{\,900}^{1100} \,{\rm L}_\odot$
these correspond to 
% $10\,{\rm M}_\odot$ and $3.9\,{\rm R}_\odot$
$9.7\,{\mypm}\,_{0.6}^{0.3}\,{\rm M}_\odot$ and $3.9\,{\mypm}\,_{0.2}^{0.1}\,{\rm R}_\odot$
\citep{1996MNRAS.281..257T}, respectively.
The accreted mass is inferred from $E_{\rm acc}\,{=}\,GM_*M_{\rm acc}/R_*$, where G is the gravitational constant, $M_*$ is 
the stellar mass, $M_{\rm acc}$ is the accreted mass, and $E_{\rm acc}$ the released energy derived %in Sec.\,\ref{drt1}
%{\red factor 2!}\\
in the previous section. Here it is implicitly assumed that the potential energy is released as well when the matter eventually reaches the protostar.

\begin{figure}
    \centering
	\resizebox{\hsize}{!}{\includegraphics{acc.png}}
	\caption{Dependence of the derived accretion rate on the average luminosity increase ${<}\Delta L^{\rm acc}{>}$ and assumed ZAMS stellar mass for the parameter range bracketing MM1. The horizontal dashed line corresponds to ${<}\Delta L^{\rm acc}{>}$ and the light gray region marks its 1$\sigma$ uncertainty while the vertical dashed line and the darker gray region indicate the stellar mass and its uncertainty. The values of the accretion rate are indicated.
	}
 \label{fig:acc}
\end{figure}

% {\tt\red for the accretion rate, the duration of the burst period is relevant, NOT the duration of the energy release!}\\
Finally, the mass accretion rate will be obtained from $\dot{M}_{\rm acc}{=}M_{\rm acc}/\Delta t_{\rm acc}$. We have to emphasize here that, generally, the duration of the enhanced accretion $\Delta t_{\rm acc}$ will be shorter than that of the elevated emission $\Delta t$ which has to be used to come up with an overall burst energy estimate. Since the maser excitation is due to MIR dust emission, the total maser flux can be taken as a proxy for the accretion strength  (see Sec.\,\ref{bd}). From the rise and fall of the maser light curve (MacLeod et al., in prep.; Yonekura et al., in prep.) an effective duration $\Delta t_{\rm acc}$ of about two months can be derived with a presumed uncertainty range of ${\pm}$5\,days.
%\\ {\tt\red errors still missing}\\
With the above quantities we obtain %$M_{\rm acc}{=}2.6\times10^{-6}\,$M$_{\odot}$, and $\dot{M}_{\rm acc}{=}10^{-4}$M$_{\odot}$$\rm yr^{-1}$.
$M_{\rm acc}\,{=}\,3.1\mypm^{1.2}_{0.9}\,{\times}\,10^{-4}\,{\rm M}_{\odot}$, and $\dot{M}_{\rm acc}{=}1.8\mypm^{1.2}_{1.1}\,{\times}\,10^{-3}\,{\rm M}_{\odot}\rm yr^{-1}$,
%\textcolor{red}{
where the errors are dominated by the error of 
%\textcolor{red}{
$\Delta t_{\rm acc}$ and
%} 
$E_{\rm acc}$. 
%}
Using the ZAMS mass-radius relation the range of the accretion rate for a given average luminosity increase is shown in Fig.\,\ref{fig:acc}.


However, since MM1 is likely in an earlier evolutionary stage, preceding the ZAMS, the above assumption may not hold. A different and presumably more realistic approach is possible using the
%most probable 
stellar radius from the RT modeling of the pre-burst SED together with the stellar mass of $12\,{\pm}3\,{\rm M}_\odot$ derived from the kinematic model of the spiral-arm accretion flows \citep{2020NatAs.tmp..144C}. This leads to
$M_{\rm acc}\,{=}\,5.3\mypm_{4.4}^{11.1}\,{\times}\,10^{-4}\,{\rm M}_{\odot}$, and $\dot{M}_{\rm acc}\,{=}\,3.2\mypm^{5.4}_{3.0}\,{\times}\,10^{-3}\,{\rm M}_{\odot}\rm yr^{-1}$. The large positive error range is mainly due to the corresponding large uncertainty of the stellar radius.
%\textcolor{red}{only + error dominated by R, for - R and t both dominant}.
%\textcolor{red}{These huge errors come mainly from $R_*=8.4\mypm^{15.6}_{5.5}$. Errors of M*, R*, Eacc are included, but not for tacc.}
To put this into perspective, during its
short 
burst G358 MM1 consumed about 180 Earth masses. Notably, because of the small disk mass, the accreted fraction represents 16\% of the total. This raises the question whether the lightweight disk is a stable or transient feature.

%\textcolor{red}{include dt 5d, consummation of 16\% of disk mass, disk possibly transient. } 



\begin{figure*}
\centering
%    \sidecaption
%	\includegraphics[width=\columnwidth]{G358_XZ-T2b.png}
%	\includegraphics[width=17cm]{G358_XZ-T2wb.png}
	\includegraphics[width=\textwidth]{G358_XZ-T3wb.png}
\caption{Temperature distribution
%- (upper) and density-distribution (lower panel) 
of the mean model in the x-z plane (innermost part of first quadrant) for the pre-burst ({\bf left}), burst- ({\bf center}) and post-burst epochs ({\bf right}).  The orange and red lines enclose the temperature range from 94\,K (orange) to 120\,K (red) during the pre- and burst epochs, respectively. The white contours mark gas particle volume densities of % $\rho_{\rm dust}\,{=}\,
%[1, 0.5, 0.3, 0.2] in units of
%$\rm 8.8\,{\times}\,10^{-18}\,g\, cm^{-3}$, where the latter corresponds to a gas particle density of 
$n_{\rm H_2}\,{=}\,[0.2,0.3,0.5,1]\,{\times}\,10^{8}\,\rm cm^{-3}$ which decrease with increasing z. The vertical solid black line indicates the outer radius of the disk. The dashed black lines mark the radius of the maser ring from the first and 4th epoch of the VLBI observations \citep[Burns et al., in prep.;]{2020NatAs...4..506B}. The length of the black bar corresponds to 50\,mas.
    %with the relation $\rho\,{=}\,\frac{m_{H_2}}{NA}\,{\times}\,\frac{n_H}{gas2dust}$
	}
 \label{fig: T, rho}
\end{figure*}


\section{Methanol maser relocation}

%\begin{figure}
%\centering
%	\includegraphics[width=\columnwidth]{G358_T-profile_mean-model.png}
%	\caption{The plot shows dust temperature profiles along the disk mid-plane (solid lines) and along the symmetry-axis (dotted lines) for the mean model (blue) and the same model but with a luminosity increased according to the mean post- and burst luminosities (green and red). The solid black line denotes a temperature of $94K$ (desorption temperature of methanol), the solid gray line show the mean disk outer radius together with the $1 \sigma$-confidence interval (dashed). The optical depth is highest at the disk mid-plane and lowest at the symmetry-axis of the system. 
	%{\tt\red Thus the given temperatures are lower/upper limits. ?} \textcolor{green}{Können wir streichen, ich wollte darauf hinaus, dass sich T(r=r0) zwischen den Kurven 'bewegt'.}
%	}
 %\label{fig:T-profile}
%\end{figure}

It has been emphasized that methanol masers play a crucial role in identifying accretion bursts from MYSOs. The maser activity of G358 during its burst was extraordinary and unique in several aspects \citep{2019ApJ...876L..25B, 2019ApJ...881L..39B, 2019MNRAS.489.3981M}.
% \\{\tt\red references}\\
The excitation of new maser spots at larger distance has been observed for the first time during the accretion burst of S255IR-NIRS3 \citep{2017A&A...600L...8M}. Notably, for G358 a ring-like propagation of maser spots, likely excited by the heat wave due to the burst, has been witnessed \citep{2020NatAs...4..506B}. Further evidence for spatial changes of the G358 maser distribution during and after the burst has gained
%by VLA observations
(Bayandina et al., in prep.). 

The presence of both methanol in the gas phase and the proper IR radiation are two of the major requirements for maser excitation. 
%{\red
While cosmic ray sputtering of dust grain mantles can also release methanol to the gas phase \citep{2020A&A...634A.103D}, thermal desorption due to grain heating will be the dominant process near the MYSO. It 
%}
%The heating of the dust grains 
%at larger distances 
%in the circumstellar environment leads to thermal desorption of frozen methanol which 
requires temperatures of at least 94\,K \citep{2018MNRAS.473.1967L} while the maximum desorption rate is reached at about 120-125\,K \citep{2004MNRAS.354.1133C}. 
%At the same time
%{\red
For such temperatures,
%}
the peak flux of the thermal IR radiation from the warm dust is in the proper wavelength range needed to excite these masers \citep{2002IAUS..206..183O, 2005MNRAS.360..533C}. 
The RT models of MM1 during the pre-, burst, and post-burst epochs described in Sec.\,\ref{drt1} yielded spatial dust temperature distributions which can be used to address which regions of the circumstellar environment are potential sites of Class II 6.7\,GHz methanol masers, and how they change due to the burst.
%Fig. \ref{fig:T-profile} shows the temperature at the disk mid-plane (solid lines) and at the symmetry-axis of the system (dashed lines) for the mean model (i.e. the model, where all parameters are set to the mean values given in Table \ref{tab MM1 fit}). There are three curves, where in blue the source luminosity is the same as the mean pre-burst luminosity and in green and red it is increased according to the mean luminosity increase during the post- and burst. A temperature of 94\,K is indicated with the solid black line. The gray lines show the mean and the $1 \sigma$-confidence interval of $r_{\rm max}^{\rm disk}$. Since the temperature is lowest at the disk mid-plane and highest along the symmetry-axis, the curves give the minimal and maximal distance, where methanol may appear in the gas-phase. 

% The dust temperature for the best burst models at 550\,au is around 70-115\,K. 


Figure \ref{fig: T, rho} shows the central part of temperature distribution of the mean model in the first quadrant of the x-z plane for the pre-burst (left), burst (middle) and post-burst (right) epochs. The white lines marks the following gas densities $n_{\rm H_2}\,{=}\,[1, 0.5, 0.3, 0.2]\,{\times}\,10^{8}\,\rm cm^{-3}$ (density decreases with z). To transform the dust densities from the RT-simulation to the above given gas densities, we use the relation $\rho\,{=}\,{m_{\rm H_2}}{N_A^{-1}}\,{\times}\,n_{\rm H_2}\gamma^{-1}$, where $m_{\rm H_2}$ is the molar mass of H$_2$, $N_A$ is the Avogadro constant and $\gamma$ is the dust to gas ratio introduced in Sec.\,\ref{drt3}. Because of its low mass, the disk, which is embedded in the rotationally flattened envelope, cannot be  easily recognized. The vertical solid black line indicates the outer radius of the disk.

For gas densities exceeding $\rm 10^{8}\,cm^{-3}$ the maser brightness drops rapidly
due to collisional de-excitation \citep[Fig. 2,]{2005MNRAS.360..533C}. Therefore, within the densest regions (innermost contour - disk mid-plane, envelope at the centrifugal radius) the excitation of Class II methanol masers is basically ruled out. 

The orange and red lines of Fig. \ref{fig: T, rho} indicate temperatures of $\rm 94\,K$ (orange, minimum temperature for thermal methanol desorption to occur) and $\rm 120\, K$ (red, optimum temperature for methanol desorption) during the pre-burst, burst and post-burst, respectively. At each epoch the temperatures right of the solid orange line are too low for desorption of methanol, i.e., it remains bound within icy dust grain mantles. Thus, these lines represent the methanol snow line beyond which masers cannot occur.
%The regions, where methanol masers can be stimulated are above the solid black line and left of the solid blue line. 
Due to the heating by the burst 
% the location, where methanol 
%is apparent in the gas phase 
it will be shifted outward
%(at maximum to the solid red line).
and move inward again once the disk started to cool after the burst ceased.

During the pre-burst stage (Fig.\,\ref{fig: T, rho} left), possible maser sites are likely confined to a region which originates in the disk around ${\approx}\,300$\,au and stretches along the surface of the cavity wall. The situation is different for the burst epoch (Fig.\,\ref{fig: T, rho} center). As expected, the methanol snow line moved outward and is located at about 1000\,au. At the same time the region below the disk/envelope surface has become warmer and presumably got enriched with gaseous methanol. These conclusions from the RT modeling can be compared to the multi-epoch VLBI observations of the expanding maser ring \citep[Burns et al., in prep.;]{2020NatAs...4..506B}.
The vertical dashed black lines mark the circumference of the maser ring from the first and 4th VLBI epoch. The first epoch observations were obtained already two weeks after the flare started. It is reasonable to assume that during the very early flare rise the excitation conditions were not too far off the pre-burst case. In fact the location of the maser circumference during the first epoch is within the region where phase transition from the solid to the gaseous methanol seems to occur in the pre-burst state. The data of the fourth VLBI epoch was taken about three weeks after the first FIFI-LS observations ('burst' epoch). The maser circumference at that time is confined to the desorption temperature interval for regions not too far off the disk plane. The fact that the extent of the maser ring matches the expected maser positions for {\em both} epochs suggests that our RT modeling, although static, nevertheless describes major changes of the circumstellar environment due to the burst properly.

Moreover, Fig.\,\ref{fig: T, rho} seems to indicate that at the first VLBI epoch, the maser sites were likely close to the interface between the outflow lobe and the disk/envelope,
%{\red
i.e., in a region of relatively low optical depth. The subliminal maser propagation speed reported by \citet{2020NatAs...4..506B} implies substantial optical depths. This suggests that the masers likely propagated into the disk rather than on its surface.
%}
%while at the fourth epoch they might have been more embedded in the latter.  If this actually holds, it would imply that the optical depth toward the maser locations may have been different - lower at the first and higher at the fourth epoch. 
If so it could be expected that the ring expansion slowed down over time. This conclusion is fairly speculative and its verification requires time-dependent RT simulations. 

The right panel of Fig.\,\ref{fig: T, rho} indicates the readjustment of temperature distribution after the burst. Thus, the desorption range moved inward and is basically in between those for both pre-burst and burst.

%\textcolor{red}{at the position of the Cavity wall/'upper Envelope/Disk' the dashed black lines hit the 120K-isothermal-lines (desorption optimum) -> possible sites of the maser excitation}

%{\red So the ring might not be too far off the methanol snow line.} %\textcolor{green}{For the mean model: the radius, where a temperature of 94\,K is reached shifts from 550\,au to 1200\,au.}
%{\red
Another important parameter for the maser excitation is the specific column density, $N_{\rm M} {\Delta V}^{-1}$, where $N_{\rm M}$ is the methanol column density along the line of sight and ${\Delta V}$ the velocity range of the molecules. 
% Der Satz ist nur sinnvoll wenn dann noch etwas in punktospez. Säulendichte kommt
%According to \citet{2005MNRAS.360..533C} at least $3\,{\times}\,10^{11}\,\rm cm^{-3}\,s$ is required. 
The low inclination $i$ obtained from our RT modeling and by \citet{2020NatAs.tmp..144C} implies that $\Delta V$ is only slightly larger than thermal line-width since the radial velocity component of the orbital motion will be small. This increases the prospects for maser to occur.
%}

When assessing these results it should be kept in mind that the circumstellar environment of G358 is almost certainly more structured than our RT model. Thus, sight lines of lower optical depth as well as density fluctuations may leave an imprint on the actual distribution of the maser spots.

\section{Wavelength and time dependence of the flux variation}\label{wtd}


The RT results for G358 can be used to address the question which wavelength range is suited best to detect the flux increase induced by a MYSO burst. For this purpose, the mean model SEDs for burst and post-burst were normalized by the pre-burst model SED. It has to be kept in mind that dividing the flux values cancels the extinction correction. The result is shown in the upper panel of  Fig.\,\ref{fig:wfr}. It can be seen that the peak values of the relative flux $m$ exceed the luminosity gains derived in the previous section. For our MYSO models the largest relative flux increase occurs around 10\,$\mu$m. This agrees with the results of \citet{2019MNRAS.487.4465M} for similar luminosity gains. %According to these authors
For their low-mass YSO models, the maximum flux ratio shifts toward the FIR for stronger accretion bursts.
%Interestingly, the best burst model seems to indicate that the silicate feature at {$\approx$}\,10\,$\mu$m appears in emission during the burst. While this would not have been observable for G358 because of the high interstellar extinction, it might be considered as another tracer for MYSO bursts. 
The least relative flux increase occurs in the (sub)mm. This is due to the fact that the emission in this part of the spectrum arises from the Rayleigh-Jeans tail of the Planck function, where the spectral radiance only linearly depends on temperature. Since the temperature increase of the outer circumstellar regions is lower than for the inner ones, their flux increase is accordingly smaller. 

\begin{figure}
	\centering
%	\resizebox{\hsize}{!}{\includegraphics{flussverhältnis.png}}
    \includegraphics[width=\hsize]{flussverhältnis.png}
	\caption{({\bf top}) Wavelength dependence of the relative flux increase for the burst (red) and post-burst (green) epochs for mean model. Markers denote grid points.
	({\bf bottom}) Wavelength dependence of the speed of the relative flux increase $m$ for the pre-burst to burst (blue) and burst to post-burst (green) periods. The black solid lines mark linear fits to $\lambda\,{>}\,10\mu m$, the dashed vertical lines indicates the wavelengths used by \citet{10.1093/mnras/staa1254}. The black dots mark the values for $m$ used to compare our models to \citet{10.1093/mnras/staa1254}.	}
    \label{fig:wfr}
\end{figure}

A recent study by \citet{10.1093/mnras/staa1254}, based on MIR (NEOWISE W2@4.6\,$\mu$m) and (sub)mm (SCUBA-2@850$\mu$m) variability of deeply embedded protostars, shed first light on the the wavelength dependence of the  rise/drop speed of the relative flux $m$. From our SED fits, we can assess this issue in a `semi-empirical' manner for the whole wavelength range of the SED. The relative rise/drop speeds are obtained for the pre- to the burst (Pr$\to$B) and the burst to post-burst (B$\to$Po) epochs, respectively. The results are presented in the lower panel of Fig.\,\ref{fig:wfr}. The two curves show $m(\lambda)=|\,log(F_{\rm x}(\lambda)/F_{\rm y}(\lambda))\,{\times}\,{\Delta t}^{-1}\,|$ for each wavelength of the modeled SEDs. The indices $x, y$ denote the respective epoch pairs and $\Delta t$ is the
%defined as the logarithm of the flux ratio for the respective epoch pair, divided by the 
corresponding epoch difference (in years). 
%This is a measure for the speed of the flux change which, obviously, depends on wavelength. 
The upper curve holds for the flux rise while the lower one for the flux decrease. Since we do not know exactly when the rise started, the upper curve may be shifted while its slope is unaffected from the actual date.

With only two wavelengths at hand, \citet{10.1093/mnras/staa1254} were not able to infer the functional form of $m(\lambda)$ and had to assume a proportionality such that $m({\rm 4.6\mu m})\,{=}\,\eta\, m({\rm 850\mu m})$. From objects with the most significant variability at both wavelengths they derived $\eta\,{=}\,5.53\,{\pm}\,0.29$. This can be compared to the $\eta$ values based on $m$ at 4.6 and 850\,$\mu$m from our results (black dots in Fig. \ref{fig:wfr}). These amount to $\eta_{\rm Pr \to B}\,{=}\,4.48$ and $\eta_{\rm B\to Po}\,{=}\,5.88$ for rise and drop, respectively. The values from our models are similar, although $\eta_{\rm Pr \to B}$ is somewhat lower.
% erst mal auskommentiert, weil sie andere passive Scheibenmodelle anders verwenden.
%\textcolor{red}{For PassIVe DiskS IT iS closer tO 4, SeC 6.3 FIg10} 

More importantly, our approach allowed us to derived $m(\lambda)$ for the whole range of the SED. The corresponding curves show that
%For  wavelengths greater than 10\,$\mu$m (silicate feature)
beyond $\lambda\,{\gtrsim}\,10\,\mu$m, $log(m)$ depends approximately linear on $log(\lambda)$. Thus, the wavelength dependence $m(\lambda)$ actually resembles a power law. The black lines show the respective fits to both epoch pairs. They yielded the following relations for $m(\lambda)$: $m_{\rm Pr \to B}\,{\propto}\, \lambda^{-0.42 \pm 0.01}$ and $m_{\rm B \to Po}\,{=}\, (1.16 \mypm^{0.04}_{0.03})\times \, \lambda^{-0.43 \pm 0.01}$, for the rise and drop, respectively. 
The spectral indices of both relations agree within the errors, suggesting that the sense of the flux variability does not affect the speed of the relative flux change.
%This relations are power-laws, which is different to \citetads[Eq. 3 and 4,]{10.1093/mnras/staa1254}, where a linear relation between $m_\lambda_1(t)$ and $m_\lambda_2(t)$ was assumed. -> the parameter there is t (where we also assume lin behaviour), not lambda
%\textcolor{red}{According to \citet{10.1093/mnras/staa1254} $\rm log(F_{\rm 4.6}(t))$ is proportional to $\rm log(F_{\rm 850}(t))$ with the proportionality constant $\eta$ (Formel aus dem abstract, better give eq 3, 4?).}
%Since for wavelengths lower than $10\mu m$, the above obtained power-law does not hold, we took the values from our models directly instead of obtaining them from our relation for $m$. 
% based on the spectral flux ratios for the burst- and post-burst epochs we can also assess the wavelength dependence of the flux decay which is related to the radiative cooling. The relative flux change per unit time is shown by the XXX line. Obviously, it is fastest in the MIR and slowest in the (sub)mm. This has also been found by \citet{10.1093/mnras/staa1254} when studying the MIR and (sub)mm variability of deeply embedded protostars.
% Pena+ (2020) suggest that heating/cooling have same time scales 
%\textcolor{green}{In order to compare our fit to \citet{10.1093/mnras/staa1254} we use our fits to obtain m-values for the wavelengths W2 ($4.618\mu m$) and SCUBA ($850 \mu m$) with this we compute $\eta$ from their Eq. 3. 
%We obtain $\eta_{\rm p\rightarrow b}\,{=}\,7.26\,{\pm}\,2.12$ for the heating (from pre- to burst) and $\eta_{\rm b\rightarrow pb}\,{=}\,8.54\,{\pm}\,3.08$ (for the cooling from burst to post-burst). %and. For comparison the values from \citetads[Fig. 7]{10.1093/mnras/staa1254} are $\eta=[10.65\pm 2.81,\, 6.18\pm 1.22,\, 6.18\pm 0.65]$ for the cooling source 52 in OMC2/3 (HOPS 383), the cooling source 1 in NGC2068 (HOPS 358) and the heating source 2 in Serpens Main (EC53) respectively. For the W2-band the Fit lays over the model (using the modelflux instead, would lead to lower values for $\eta$).}
%For comparison, the value from \citet{10.1093/mnras/staa1254} is $\eta\,{=}\,5.53\,{\pm}\,0.29$, obtained for objects with the most significant variability at both wavelengths. 
%{\red Within the errors our estimates agree with the value of \citet{10.1093/mnras/staa1254}.} 
These results indicate that a) burst induced flux variations are quicker at shorter wavelengths and b) the rise and fall of the fluxes take equal times.

Yet, a cautionary note is required here. First of all, our assessment of the correlated flux variability is based on YSO models with passive disks. The viscous energy release in an active disk leads to a temperature distribution that differs from the passive case in particular in the innermost region. This will lead to a higher value of $\eta$ as shown by \citet{10.1093/mnras/staa1254}. Moreover, we have to mention that the models used by these authors and us are based on static RT. Time-dependent RT shall be used to verify the above considerations. 

\section{Discussion}\label{disc}
In the following, particular results are evaluated with regard to their credibility and impact concerning our understanding of MYSO accretion bursts. The G358 event is compared to the previous cases. Before we address individual topics we note for completeness, that our investigation only covers the surplus dust continuum emission caused by the accretion burst. While the fraction of
%a small but perhaps non-negligible fraction of 
the energy consumed by phase transitions (sublimation of grain ice mantles, dust evaporation and dissociation of molecules)
%, H$_2$ in the first place, have to be considered. In order to account for those processes, thermo-chemical models of
is negligible in the present context, it nevertheless seriously affects the chemistry of the YSO \citep{2017A&A...604A..15R, 2019MNRAS.485.1843W}.

\subsection{Misfit of the (sub)mm fluxes}\label{misf}
%{\red
Figure \ref{fig:sed g358} shows that the best burst and post-burst fits predict (sub)mm fluxes that exceed the measured values. %see the 
The observed values, indicated by the red and green symbols at $\lambda\,{=}\,889\,\mu$m amount to 60-70\% of the respective mean burst and post-burst model.
% observed with ALMA 
Since attempts to solve this discrepancy failed or required non-plausible assumptions, e.g., on grain properties or geometry, we conclude that this mismatch indicates a deviation of the temperature distribution of the circumstellar environment from the static case. The accretion burst causes an additional inside-out heating of the circumstellar environment where inner regions try to reach a new thermal equilibrium while outer ones are still in the previous steady-state.  Time-dependent RT simulations suggest that the heating timescales 
%are longest in the \textcolor{green}{FIR/ streichen? oder for the longest wavelengths} FIR/sub-mm 
%{\red
strongly increase with wavelength
%}
\citep{2011MNRAS.416.1500H, 2013ApJ...765..133J}.  
%This implies that, in this regime, the deviation from static models (as the ones we use to model the SEDs) to a time-dependent treatment is greatest.
%}
%\textcolor{green}{
Since the MM1 SEDs are dominated by the increased FIR fluxes, their fits will, therefore, yield over-predicted (sub)mm fluxes.
%will be higher than the observed ones 

For deeply embedded YSOs, the contribution of the envelope to the thermal FIR/(sub)mm emission may be significant (or even dominant) ,e.g., \cite{10.1093/mnras/staa1254}. For about half of the best MM1 pre-burst models, the overall optical depth of the envelope exceeds by far that of the disk (see Table \ref{tab: tau_V}). In these cases the envelope produces the overwhelming fraction of the (sub)mm emission. 
%\textcolor{red}{ref to tau table, high optical depths also in the env at least for some models, Meanmodel? 
%Mention that disk probably TRANSIENT? <- steht in 'Accretion burst drivers'
%}
%While \citep{2011MNRAS.416.1500H} focuses on the disk (and especially its cooler interior regions), the over-estimation of the FIR/sub-mm flux might be also relevant for the thermal emission originating from the envelope.}
%One argument for that is, that the envelope is more extended and the radiation will reach the locations with an additional delay due to longer light travel times. The 'outer radius' of the envelope is set to the point, where its density (and temperature) has dropped to the ambient level. This occurs at $3\,{\times}\,10^6$\,au or 50 light years. However, the envelope is densest at it's centrifugal radius $r_C$, which is located at the outer radius of the disk. Therefore the delay due to longer light paths may be less relevant. 
%\textcolor{green}{
Furthermore, the heating of parts of the envelope, which are in the shadow of the disk, will be suppressed or (strongly) delayed. This effect will be especially important for systems with strongly flared disks. 
%}

%\textcolor{green}{Another effect is that YSO do have a lot of material in its outer regions. i.e. in the envelope. Assuming the burst to travel on the speed of light, it will reach the centrifugal-radius $r_C$ (which is assumed to be at the outer edge of the disk) at roughly half a month. The envelope density decreases outside $r_C$ and reaches the ambient level at about $3\,{\times}\,10^6$\,au. At a distance of 6500\,au the light needs nearly 3 months (1 burst duration). These numbers highlight, that not only the disk is non-stationary, but also parts of the envelope. This might further increase the above mentioned systematic effect.} %\textcolor{red}{To Discuss: What about temperatures, how much does the outer envelope contribute to the (sub)mm/FIR-emission compared to the disk?}
%In this sense ignoring the sub-mm burst fluxes in the static treatment may be justified.
Not only the heating time scales have an influence on the burst-SED, but also the characteristics of the burst itself. Short/weak bursts might be not strong or long enough to heat the %\textcolor{green}{
whole circumstellar environment which is probably the case for G358. 
%}.
%entire disk as well as the protostellar envelope. 
This will lead to an overestimation of the sub-mm fluxes when using stationary models as well. Because of that
%We don't use 
fluxes beyond $889\,\mu $m were not included to fit the MM1 burst SED in order to avoid their systematic overestimation induced by our method.

%It has to be emphasized that only % the ALMA flux and 
%the upper limits refer to MM1. 
%All other values represent integral fluxes for the G358 complex. This implies that the pre-burst luminosity for MM1 is overestimated, leading to an estimate of the luminosity increase smaller than the actual value. However, MM1 is by far the brightest (sub)mm source \citep{2019ApJ...881L..39B}. Thus it is very probable that it also dominates the FIR emission of G358. 
%\textcolor{red}{{\tt\red old stuff, delete} While the MM3-subtracted SED is the best approximation for the SED of MM1, given the present data, it still contains contributions from the remaining objects of the star forming region. Since none of them shows up in the NIR/MIR they seem to be deeply embedded as well and, thus, contribute to the FIR/(sub)mmm emission. This becomes obvious by summing up, e.g., their ALMA 889\,$\mu$m emission which is similar to that of MM1. Thus, the $\Delta L_{\rm acc}$ derived above has to be considered as lower limit.}

\subsection{Uncertainty of the accretion parameters}
%Moreover, there is additional uncertainty for both accretion rate and accreted mass. According to the above consideration the actual pre-burst luminosity implies smaller ZAMS $M_*$ and $R_*$ values than adopted in our analysis which partially counterbalances a larger $\Delta L_{\rm acc}$.

It was mentioned above that the ZAMS values for $R_*$ might not apply since the high accretion rates during the growth of a massive protostar will temporarily bloat its radius which could then be many times larger than that of a main sequence star (\citealp{2009ApJ...691..823H, 2010ApJ...721..478H}).
%{\red
Recent simulations suggest that bloating is even intensified by accretion bursts \citep{2019MNRAS.484.2482M}. 
As a consequence of bloating,
%}
more mass needs to be accreted to produce the observed luminosity increase. 
%{\tt\red is there a diagnostic for MYSOs to infer accretion rates?}
In the cold disk accretion model of \citet{2010ApJ...721..478H} the protostellar swelling occurs in the mass range between 5-9\,M$_\odot$. Since the mass of MM1 is likely of that order, the supposition of being bloated is perhaps valid. If so, the accretion rate may be elevated as it is proportional to the stellar radius. Unfortunately, due to the high extinction of deeply embedded protostars it is hard to derive their surface gravity from photospheric absorption lines because of strong veiling. 
% While this has been achieved for a Class~0 low-mass YSO \citet{2018ApJ...862...85G} a similar result for MYSOs is still lacking.
Only recently this has been achieved for the first time. The MYSO G015.1288--00.6717 shows an A-type spectrum with a relatively low surface gravity, pointing to a bloated protostar \citep{2017MNRAS.472.3624P}.

For that reason,
%It is necessary to emphasize that 
caution is required for what concerns the comparison of stellar-related burst properties. For a MYSO in a relatively evolved stage, like S255IR-NIRS3 which shows a flat pre-burst SED, the derivation of an accretion rate using the ZAMS approach seems to be justified. However, its application to MYSOs in earlier stages like G358 or NGC6334I-MM1 is questionable.
% Deuterium burning sets in at 0.3M_sun in case of disk accretion Hosokawa+
% the luminosity of these objects is entirely dominated be accretion.
In addition, the ZAMS approach has its own challenges because, unlike low-mass stars, massive stars arrive on the ZAMS while accreting. Therefore, their pre-burst luminosity includes a likely small but unknown fraction of accretion luminosity. For low- and intermediate mass-stars, accretion rates can be derived from emission-line tracers (e.g., \citealp{2011A&A...535A..99M}) which, however, might be only exceptionally seen in scattered light from MYSOs \citep{1995ApJ...448..832N}. In any case the burst energy $E_{\rm acc}$ provides a good proxy on the energetics of the event. In this regard, the values for G358-MM1, S255IR-NIRS3, and NGC6334I-MM1 of %$(2.7\mypm^{0.56}_{0.84}){\times}10^{37}$\,J
%$(5.7\,{\mypm}^{\,1.9}_{\,1.5})\,{\times}\,10^{37}$\,J
$2.9\mypm_{0.8}^{1.1}\,{\times}\,10^{38}$\,J
(see Sec.\,\ref{drt1}), $1.2\pm{0.4}\,{\times}\,10^{39}$\,J \citep{2017NatPh..13..276C}, and $8\,{\times}\,10^{38}$\,J \citep{2017ApJ...837L..29H} already indicate a range of about an order of magnitude. Since the luminosities of S255IR-NIRS3 and NGC6334I-MM1 were still elevated when the aforementioned papers appeared, their final numbers will be even higher.

%{\red
\subsection{Burst strength bias}
It is well established that young massive stars have a high stellar multiplicity \citep{2007ARA&A..45..481Z, 2019MNRAS.484..226P}. Given their distances, high spatial resolution obtained by interferometric observations is required to resolve the individual components of MYSOs like G358 or to prove their single nature. For less embedded objects this can be achieved with the VLTI in the thermal IR, while for deeply embedded sources (sub)mm interferometers like NOEMA or ALMA have to be used. 
%Such a capability does not exist yet in the FIR
FIR interferometry, which requires groups of satellites in space, does not yet exist, although it has been envisaged \citep{2020AdSpR..65..831L}. Keeping in mind that MYSOs emit the bulk of their energy in the FIR, the coarse spatial resolution in the FIR implies that in most if not all cases, their luminosities consist of individual contributions. Thus, relating the luminosity increase caused by an accretion burst of single source to the total pre-burst luminosity only yields a lower limit to the burst strength. In case of G358 the luminosity gain derived from the graybody fits of pre- and burst total fluxes amounts to 2.5 while the RT results based on the SED decomposition (see Sec.\,\ref{sedc}) yielded a value of 4.7 for MM1.

%}

\subsection{Burst duration and temporal evolution of the IR emission}\label{bd}
Class II methanol masers are excited by MIR thermal emission from warm dust grains with temperatures exceeding 100\,K
\citep{2002IAUS..206..183O, 2005MNRAS.360..533C}. Regardless of the details of the enhanced accretion, the increase of released energy will in all likelihood cause the evaporation of dust grains at the inner boundary of the disk and thus shift its rim outward. Since the radiative transfer will adjust the temperature profile accordingly, the MIR-emitting region moves toward a larger radius.
Due to its larger surface area
the emission increases and that of the masers as well. Thus, the rise and fall of the maser emission is coupled to the accretion variability through the dust MIR continuum emission. First direct evidence for this relation has been found by us in the case of S255IR-NIRS3 \citep{hansfest}. Since the maser flare of G358 lasted only for a few months %\citep{2020NatAs.tmp..144C} 
(MacLeod et al., in prep.; Yonekura et al., in prep.), it can be concluded that the increased accretion rate dropped shortly before the end of the flare. So the question arises how does it come that elevated FIR emission is still present after 18 months past the flare peak? The current notion is that dust cooling should be quick because of the low optical depth at those wavelengths. The bulk of the FIR/(sub)mm emission comes from outer regions of the circumstellar environment. Thus, their optical depth toward the observer is low indeed. However, this is not necessarily the case for the optical depth towards the protostar where the energy is being released. An {\em effective} optical depth of a few will slow down the heating of the outer YSO environment and 'trap' burst energy in the dusty medium. The latter will be radiated away and cause elevated FIR emission for quite some time after the burst terminated. First evidence of post-burst elevated FIR emission has been found by us in the case of S255IR-NIRS3 (Stecklum et al., in prep.). G358 is another example which shows this behavior as well. 
%{\red Evidence for short and long time scales for the flux rise/drop of inner and outer regions of deeply embedded YSOs has been found recently \citep{10.1093/mnras/staa1254}, based on their secular MIR and (sub)mm variability.}


The projected linear separations between nearest neighbors of the G358 complex of a up to 10$^4$\,au \citep{2019ApJ...881L..39B} imply object sizes with corresponding light travel times of up to two months. These can easily stretch to one year or more for moderate efficient optical depths. Furthermore, it has to be kept in mind that there is a variety of sight lines with different optical depths along which outer regions will be heated. The different propagation speeds along those lines incur a broadening of the heating duration. A subluminal speed  (0.04$\dots0.08\,c$) of the heat wave during the burst has been witnessed in G358 by tracing, for the first time, the outward motion of maser spots \citep{2020NatAs...4..506B}. Subluminal speed for the relocation of maser spots was found for S255IR-NIRS3 \citep{2018IAUS..336...37S} as well. This illustrates the imprint of the optical depth on the radiative transfer of the burst energy throughout the YSO.

%on the temporal behavior of the dust continuum emission.  

\subsection{Accretion burst drivers}

YSO accretion bursts are thought to be caused by enhanced mass transfer through the circumstellar disk, triggered by various reasons, which may drive the inner disk to become active and self-luminous \citep{1996ARA&A..34..207H}. Among the possible causes, erratic infall from a protostellar envelope (cf. \citealp{2015ApJ...805..115V, 2017MNRAS.464L..90M})  seems to be favorable for a MYSO in an evolutionary stage as early as G358. Interestingly, recent high-resolution radio imaging of MM1 suggests the presence of two spiral-arm accretion flows, winding towards the protostar as traced by methanol masers \citep{2019ApJ...881L..39B, 2020NatAs.tmp..144C}. This is in line with very high-resolution ALMA imaging of high-mass protostars that provided evidence for filamentary streamers pointing onto the central sources \citep{2018arXiv180505364G, 2020A&A...634L..11J}. These might represent multi-directional accretion channels which possibly inhibit the formation of a large, steady disc at the very early stages of massive star formation. Thus, the small disk mass suggested by our best RT SED fits may be seen as a hint for the extreme youth of MM1. Possibly, its environment is not relaxed enough to allow the formation of a larger sustainable circumstellar disk. Such a disk might be prone to gravitational instability, considered to be another cause for episodic accretion \citep{2015ApJ...805..115V}. 
%This might also play a role to explain the multitude of masering transitions \citep{2019ApJ...876L..25B, 2019MNRAS.489.3981M, 2019ApJ...881L..39B, 2020ApJ...890L..22C}, excited by the burst, most of them never seen before. 
%Sensitive ALMA continuum observations at the largest baselines are required to...
Faint radio continuum emission from MM1 has been detected during the burst as well (Bayandina et al., in prep.). Whether it is variable and perhaps related to an active disk needs to be explored.

\subsection{G358 in the accretion burst context}
The G358 burst is the second one from a MYSO for which the accretion luminosity could be measured from the SED changes. This allows us to compare the derived quantities with the results obtained for S255IR-NIRS3 \citep{2017NatPh..13..276C}, which is the first step towards a statistics of MYSO burst properties. 
%This is a prerequisite to make a comparison with corresponding predictions of burst models, e.g. \citet{2019MNRAS.482.5459M} 
Assuming a ZAMS star, an accreted mass of ${\approx}\, 3.4\,{\times}\,$10$^{-3}$\,M$_\odot$ has been estimated for S255IR-NIRS3. The corresponding accretion rate amounts to
%NIRS3 This latter amounts to an  about two Jupiter masses (namely , see Methods).we infer that $\dot{M}_{\rm acc}$ is boosted to
5$\,{\pm}\,$2\,${\times}\,$10$^{-3}$\,M$_\odot$\,yr$^{-1}$. While these values will likely be revised once the comprehensive monitoring data set has been analyzed, we can assume that their order of magnitude is correct.
The accretion rate of the G358 burst almost corresponds to that of the S255IR-NIRS3 event. However, the accreted mass is about one order of magnitude smaller due to the shorter duration of the burst. Thus,
%it is obvious that, 
compared to S255IR-NIRS3, the G358 burst was a minor one.
%, since both the accretion rate and the duration were much smaller. Consequently, the accreted mass differs by three orders of magnitude.
Simulations of MYSO accretion \citep{2019MNRAS.482.5459M} suggest that minor bursts similar to that of G358 are much more frequent compared to major ones, and occur predominantly at very early stages of protostellar evolution.

While the MYSO accretion burst sample is still small, it will grow in the mid-term by following up methanol maser alerts on a regular basis. Recently, the suggestion by \citet{2019MNRAS.487.2407P} that the periodicity of methanol masers in G323.46--0.08 could be due to an accretion burst was confirmed (Stecklum et al., in prep.) using (NEO)WISE and VVV data. While this is another event which was identified a posteriori, it raised the total number of known MYSO bursts to five,
including V723~Car.
So a  thorough comparison of MYSO burst properties with corresponding models (e.g., \citealp{2019MNRAS.482.5459M}), should become possible in the foreseeable future.

% {\tt\red compare MM1 $L$ with Brogan estimate!}

% TB finishedhttps://www.overleaf.com/project/5db2ef6d1593110001893244
% It seems plausible that, in addition to stellar heating, the in-spiraling matter becomes viscously heated as well, especially close to the protostar since the dissipation rate strongly depends on the inverse radius (\cite{1974MNRAS.168..603L}, \cite{1981ARA&A..19..137P}). 
%Because the dissipation rate is proportional to the accretion rate, a severe increase of the latter
%Eventually, both effects will drive the dust temperature as high as the dust sublimation value, thus governing the location of the inner disk radius.

\section{Conclusions}\label{conc}

The SOFIA FIFI-LS observations of G358 during two epochs, supplemented by NIR imaging and supplementary archival data, and their RT analysis yield the following major results.
\begin{itemize}
    \item Increased FIR fluxes measured with FIFI-LS confirmed the accretion burst of G358 which was alerted by the maser flare. 
    %Since it could not be verified by NIR as well as (sub)mm observations
    %{\red 
    Since no excess emission associated with the burst was detected in NIR or (sub)mm observations,
    %}
    it represents the first NIR/(sub)mm-dark, FIR-loud MYSO accretion burst.
    \item By means of the SED decomposition approach actual luminosity estimates of the driving source of the accretion burst could be obtained which yielded more representative burst parameters.
    \item From the change of the luminosities at the two FIFI-LS epochs the decay-time time of the FIR excess emission could be determined. This yielded more reliable estimates of the burst energy and the derived parameters. Moreover, wavelength-dependent rates for the rise and fall of the relative fluxes could be obtained.
    %{\red
    \item The circumstellar disk of MM1 is less massive than those found for other MYSOs and possibly transient. This may be an indication of the extreme youth of the source.
    %}
    \item A comparison with previous MYSO bursts indicates a considerable range of burst characteristics. The burst of G358 was the least energetic of the small sample.
    \item The RT modeling allowed us to draw conclusions on the possible sites of maser excitation and their re-location due to the burst which seem to be supported by VLBI radio observations.
   \item As already demonstrated in the case of the S255IR-NIRS3 event, the G358 results underline once more that the FIR observation capabilities of SOFIA are of utmost importance to study the erratic growth of stars.
 
\end{itemize}


%Thus, both, (NEO)WISE photometry and astrometry, do not provide any signs of the outburst. This is in line with the reasoning on the extreme extinction drawn from the GROND observations. 

As soon as a maser parallax for G358 becomes available, a re-analysis should be performed to narrow down the uncertainties of the derived parameters which will increase their credibility.

\begin{acknowledgements}

AV acknowledge the support from the Government of Russian Federation and Ministry of Science and Higher Education of the Russian Federation (grant N13.1902.21.0039) in the part of the data analysis.
VW is supported by the by the German Aerospace Center (DLR) under grant number 50OR1718.
ACG has received funding from the European Research Council (ERC) under the European Union’s Horizon 2020 research and innovation programme (grant agreement No. 743029)
Part of the funding for GROND (both hardware and personnel) was generously granted by the Leibniz-Prize to G. Hasinger (DFG grant HA 1850/28-1) and by the Th\"uringer Landessternwarte Tautenburg.
Based on observations made with the NASA/DLR Stratospheric Observatory for Infrared Astronomy (SOFIA). SOFIA is jointly operated by the Universities Space Research Association, Inc. (USRA), under NASA contract NNA17BF53C, and the Deutsches SOFIA Institut (DSI) under DLR contract 50 OK 0901 to the University of Stuttgart.
This research has made use of the NASA/IPAC Infrared Science Archive, which is funded by the National Aeronautics and Space Administration and operated by the California Institute of Technology.
This publication makes use of data products from the Near-Earth Object Wide-field Infrared Survey Explorer ((NEO)WISE), which is a joint project of the Jet Propulsion Laboratory/California Institute of Technology and the University of Arizona. (NEO)WISE is funded by the National Aeronautics and Space Administration.
The ATLASGAL project is a collaboration between the Max-Planck-Gesellschaft, the European Southern Observatory (ESO) and the Universidad de Chile. It includes projects E-181.C-0885,  E-78.F-9040(A), M-079.C-9501(A), M-081.C-9501(A) plus Chilean data. 
%{\tt\red to be completed}
\end{acknowledgements}

% - use BibTeX with the regular commands:
%\bibliographystyle{aa} % style aa.bst
% yield clickable references
\bibliographystyle{aa_url} % style aa_url.bst

\bibliography{G358-SOFIA} % your references Yourfile.bib

%\clearpage

\begin{appendix}\label{app}
\section{Additional information}
\begin{table}
\begin{threeparttable}[b]
\caption[]{MM3-SED. The fluxes at $\lambda\,{\ge}\,889\,\mu$m, obtained with ALMA/SMA \citep{2019ApJ...881L..39B}, are MM3-fluxes, whereas the fluxes at $\lambda\,{\le}\,24\,\mu$m are total fluxes. The contribution of all the other sources, which are less evolved than MM3 at NIR wavelengths, can be neglected (as indicated in Fig. \ref{fig:sed g358 pre}). For the fit of MM3 (see Sec. \ref{drt3}) we use the total fluxes at short wavelengths. Only in the FIR/(sub)mm-regime, where the contribution of the other sources is not negligible, we use the MM3-fluxes.}
\label{tab: MM3 SED}
\begin{tabular*}{8.5cm}{@{\hspace{1.25cm}}ccc}
\hline
\noalign{\smallskip}
Wavelength & Flux & Ref.\\
$[\mu$m] & ${[\rm erg \, cm^{-2} \, s^{-1}}]$& \\
\noalign{\smallskip}
\hline
\noalign{\smallskip}
$ ~~~~1.63 $ & $1.16 \pm 0.07\,{\times}\,10^{-12}$  & 1 \\
$ ~~~~2.13 $ & $1.67 \pm 0.02\,{\times}\,10^{-11}$  & 1 \\
$ ~~~~3.55 $ & $3.64 \pm 0.02\,{\times}\,10^{-10}$  & 2 \\
$ ~~~~4.49 $ & $6.10 \pm 0.03\,{\times}\,10^{-10}$  & 2 \\
$ ~~~~5.73 $ & $9.73 \pm 0.05\,{\times}\,10^{-10}$  & 2 \\
$ ~~~7.0~ $ & $6.90 \pm 0.49\,{\times}\,10^{-10}$  & 3 \\
$ ~~~~7.87 $ & $5.30 \pm 0.04\,{\times}\,10^{-10}$  & 2 \\
$ 11.6 $ & $ 3.54 \pm 0.07\,{\times}\,10^{-10}$  & 4 \\
$ 15.0 $ & $ 6.46 \pm 0.20\,{\times}\,10^{-10}$  & 3 \\
$ 22.1 $ & $ 5.13 \pm 0.13\,{\times}\,10^{-10}$  & 4 \\
$ 23.7 $ & $ 4.49 \pm 0.13\,{\times}\,10^{-10}$  & 5 \\
$ 889~~~~ $ & $ 1.34 \pm 0.04\,{\times}\,10^{-13}$  & 6 \\
$ 1282~~~~~ $ & $ 2.43 \pm 0.10\,{\times}\,10^{-14}$  & 6 \\
$ 1420~~~~~ $ & $ 1.50 \pm 0.19\,{\times}\,10^{-14}$  & 6 \\
$ 1532~~~~~ $ & $ 1.06 \pm 0.08\,{\times}\,10^{-14}$  & 6 \\
\noalign{\smallskip}
\hline
\end{tabular*}
\tablebib{ (1)~\cite{2017yCat.2348....0M}; (2) \cite{2017yCat.2348....0M}; (3) \cite{2003A&A...403..975O}; (4) \cite{2014yCat.2328....0C}; (5) \cite{2015AJ....149...64G}; (6) \cite{2019ApJ...881L..39B}
}
\end{threeparttable}
\end{table}

\begin{table}
\caption[]{Total- and MM1-pre-burst fluxes. We obtained the MM1 flux densities used for the SED-fit in Sec. \ref{drt1} by removing the contribution from all other sources 
%in the field (including MM3) 
according to Sec. \ref{sedc}.}
\label{tab: preSED}
\begin{tabular}{p{1.25cm} ccc}
\hline
\noalign{\smallskip}
Wavelength & Total flux & Ref. & MM1 flux \\
$[\mu$m] & $[{\rm erg \, cm^{-2} \, s^{-1}}]$ & &$[{\rm erg \, cm^{-2}\, s^{-1}}]$ \\
\noalign{\smallskip}
\hline
\noalign{\smallskip}
~~~~~$2.15$\tablefootmark{*} & $5.86 \pm 0.59\,{\times}\,10^{-13}$& 1 & \\
~~$24$\tablefootmark{*} & $5.25\,{\pm} 0.53\,{\times}\,10^{-11}$  & 1 & \\
~~$65$ & $3.10 \pm 0.06\,{\times}\,10^{-9} $  & 1 & $1.55 \pm 0.03\,{\times}\,10^{-9}$ \\
~~$70$ & $3.13 \pm 0.38\,{\times}\,10^{-9} $  & 1 & $1.57 \pm 0.06\,{\times}\,10^{-9}$ \\
% $90$\tnote{**} & $1.79 \pm 0.03\,{\times}\,10^{-9}$ & \\
$160$ & $1.90 \pm 0.21\,{\times}\,10^{-9} $  & 1 & $9.50 \pm 0.52\,{\times}\,10^{-10}$ \\
$250$ & $7.82 \pm 1.04\,{\times}\,10^{-10} $  & 1 & $3.91 \pm 0.27\,{\times}\,10^{-10}$ \\
$350$ & $2.34 \pm 0.52\,{\times}\,10^{-10} $  & 1 & $1.17 \pm 0.26\,{\times}\,10^{-10}$ \\
$500$ & $4.98 \pm 2.82\,{\times}\,10^{-11} $  & 1 & $2.49 \pm 0.33\,{\times}\,10^{-11}$ \\
$850$ & $ 4.73\pm 0.15\,{\times}\,10^{-12} $  & 2 & $2.36 \pm 0.08\,{\times}\,10^{-12}$\\ 
$870$ & $4.03 \pm 0.07\,{\times}\,10^{-12} $  & 3 & $2.00 \pm 0.35\,{\times}\,10^{-12}$ \\
\noalign{\smallskip}
\hline
\end{tabular}
\tablebib{ (1)~present paper; (2) \cite{2018ApJS..234...22P}; (3) \citet{2019ApJ...881L..39B}
}
\tablefoot{
\tablefoottext{*}{Upper limit}
}

\end{table}

\begin{table}
\begin{threeparttable}[b]
\caption[]{Total- and MM1-burst fluxes: similar to Table \ref{tab: preSED} but for the burst epoch. We assume that during the burst only the luminosity of MM1 increased while all other sources remained constant. Note that we do not provide MM1 fluxes for $\lambda>890\mu m$, since the data points at theses wavelengths are discarded for the burst SED fit as discussed in Sec. \ref{misf}.}% (i.e. at their respective pre-burst-levels).}
\label{tab: burstSED}
\begin{tabular}{cccc}
\hline
\noalign{\smallskip}
wavelength & total flux & Ref. & MM1 flux \\
$[\mu$m] & $[{\rm erg \, cm^{-2} \, s^{-1}}]$ & & $[{\rm erg \, cm^{-2}\, s^{-1}}]$ \\
\noalign{\smallskip}
\hline
\noalign{\smallskip}
$ ~~~2.15 $\tablefootmark{*}& $ 9.20 \pm 0.92\,{\times}\,10^{-13}$  & 1 &\\
$ ~~3.4~ $\tablefootmark{*}& $ 8.82 \pm 0.89 \,{\times}\,10^{-11}$  & 1 &\\ 
$ ~~4.6~ $\tablefootmark{*}& $ 2.93 \pm 0.30\,{\times}\,10^{-10}$  & 1 &\\
$ 52.0 $& $ 5.51 \pm 0.56\,{\times}\,10^{-9}$  & 1 & $ 4.85 \pm 0.49\,{\times}\,10^{-9}$ \\
$ 54.8 $& $ 6.94 \pm 0.70\,{\times}\,10^{-9}$  & 1 & $ 6.12 \pm 0.62\,{\times}\,10^{-9}$ \\
$ 60.7 $& $ 6.20 \pm 0.62\,{\times}\,10^{-9}$  & 1 & $ 5.04 \pm 0.51\,{\times}\,10^{-9}$ \\ 
$ 87.2 $ & $ 7.48 \pm 0.75\,{\times}\,10^{-9}$  & 1 & $ 5.35 \pm 0.54\,{\times}\,10^{-9}$ \\ 
$ 118.6~ $ & $ 6.45 \pm 0.65\,{\times}\,10^{-9}$  & 1& $ 4.59 \pm 0.46\,{\times}\,10^{-9}$ \\ 
$ 124.2~ $& $ 8.97 \pm 0.90 \,{\times}\,10^{-9}$  & 1  & $ 7.22 \pm 0.72\,{\times}\,10^{-9}$ \\ 
$ 142.2~ $  & $ 5.26 \pm 0.53\,{\times}\,10^{-9}$  & 1 & $ 5.25 \pm 0.53\,{\times}\,10^{-9}$ \\ 
$ 153.3~ $ & $ 5.29 \pm 0.53\,{\times}\,10^{-9}$  & 1 & $ 4.08 \pm 0.41\,{\times}\,10^{-9}$ \\ 
$ 162.8~ $  & $ 5.47 \pm 0.55\,{\times}\,10^{-9}$  & 1& $ 4.41 \pm 0.45\,{\times}\,10^{-9}$ \\ 
$ 186.4~ $& $ 4.58 \pm 0.46 \,{\times}\,10^{-9}$  & 1 & $ 3.82 \pm 0.39\,{\times}\,10^{-9}$ \\ 
$ 889~~~ $ & $ 3.81 \pm 0.11\,{\times}\,10^{-12}$  & 2 & $ 1.72 \pm 0.07\,{\times}\,10^{-12}$ \\ 
$ 1282~~~~ $ & $ 4.21 \pm 0.03\,{\times}\,10^{-13}$  & 2 & \\ 
$ 1420~~~~ $ & $ 2.74\pm 0.28\,{\times}\,10^{-13}$  & 2 & \\ 
$ 1532~~~~ $ & $ 1.88\pm 0.01\,{\times}\,10^{-13}$  & 2 & \\ 
\noalign{\smallskip}
\hline
\end{tabular}
\tablebib{ (1)~present paper; (2) \citet{2019ApJ...881L..39B}
}
\tablefoot{
\tablefoottext{*}{Upper limit}
}
\end{threeparttable}
\end{table}

\begin{table}
\begin{threeparttable}[b]
\caption[]{Total- and MM1-post-burst fluxes): similar to Table \ref{tab: preSED} but for the post-burst. We assume that only the luminosity of MM1 has changed while all other sources remained constant.} % (i.e. at their respective pre-burst-levels).}
\label{tab: postSED}
\begin{tabular}{cccc}
\hline
\noalign{\smallskip}
wavelength & total flux & Ref.  & MM1 flux  \\
$[\mu$m] & $[{\rm erg \, cm^{-2} \, s^{-1}}]$ & & $[{\rm erg \, cm^{-2}\, s^{-1}}]$ \\
\noalign{\smallskip}
\hline
\noalign{\smallskip}
$ 118.6 $ & $ 5.17 \pm 0.52\,{\times}\,10^{-9}$  & 1& $ 3.55 \pm 0.36\,{\times}\,10^{-9}$ \\ 
$ 124.2 $& $ 7.70 \pm 0.77\,{\times}\,10^{-9}$  & 1  & $ 5.55 \pm 0.56\,{\times}\,10^{-9}$ \\ 
$ 142.2 $  & $ 4.74 \pm 0.48\,{\times}\,10^{-9}$  & 1 & $ 3.28 \pm 0.33\,{\times}\,10^{-9}$ \\ 
$ 153.3 $ & $ 4.35 \pm 0.44\,{\times}\,10^{-9}$  & 1 & $ 3.09 \pm 0.31\,{\times}\,10^{-9}$ \\ 
$ 162.8 $  & $ 3.01 \pm 0.31\,{\times}\,10^{-9}$  & 1& $ 1.90 \pm 0.19\,{\times}\,10^{-9}$ \\ 
$ 186.4 $& $ 2.53 \pm 0.26 \,{\times}\,10^{-9}$  & 1 & $ 1.73 \pm 0.18\,{\times}\,10^{-9}$ \\ 
$ 889~~ $ & $ 3.81 \pm 0.11\,{\times}\,10^{-12}$  & 2 & $ 1.72 \pm 0.07\,{\times}\,10^{-12}$ \\ 


\noalign{\smallskip}
\hline
\end{tabular}
\tablebib{ (1)~present paper; (2) \citet{2019ApJ...881L..39B}
}
\end{threeparttable}
\end{table}

\begin{figure*}
%\centering
    \sidecaption
%	\includegraphics[width=\columnwidth]{G358_ap3_modelpool_obs.png}
	\includegraphics[width=12cm]{G358_ap3_modelpool_obs.png}
	\caption{Model pool (in grey) showing all SEDs, used to fit burst and post-burst epochs. The observed fluxes from both epochs are plotted in red (burst) and green (post-burst). They are well covered by the model SEDs (except for the burst data points at $163$ and 186\,$\mu$m, as discussed when introducing the model pool in Sec. \ref{drt1}). The dashed black line indicates the mean model, as obtained from the pre-burst fit alone. The NIR/MIR fluxes of that model have been used to set a lower limit to the post-burst, which was only observed in the red channel of FIFI (at $\lambda\,{\ge}\,118\,\mu$m).} 
 \label{fig:sed g358 modelpool}
\end{figure*}


\clearpage

\begin{sidewaystable}
\vspace*{9cm}
\caption[]{MM3 SED fit results: The table contains the full parameter set 
%{\red
as well as the derived luminosities (see Sec.\,\ref{rta})
%}
for the 10 best models, together wit their respective $\goodchi^2$-values. The weighted mean and standard deviation $\sigma$ is given in the bottom. The confidence interval in the case of log-sampled values extends from $\frac{x}{\sigma}$ to $x \,{\times}\, \sigma$, where for lin-spaced values it is $x \pm \sigma$ as usual. All masses (and densities) are given in unit dust mass, where we use the Weingartner \& Draine (2001) Milky Way grain size distribution A for $RV=5.5$. 
Note that, some models appear more than ones, but for different inclinations. In our analysis we thread them as different models. The envelope inner radius is set to $r_{disk}^{min}$, its centrifugal radius $r_C$ is set to $r_{disk}^{max}$. All models extends to an outer radius, where density and temperature have dropped to the ambient level. 
%The third header line indicates whether the parameter space is linearly (lin) or logarithmically (log) sampled.
}
%features an ambient medium with a density of $\rho_0^{amb}=110^{-23} \frac{g}{cc}$ and a temperature of $T_{amb}=10K$, that extends all the way to the grid.}
%(i.e. to $r_{max}^{envelope}$). Outside $r_C$ the envelope density drops. With the ambient medium we basically set a lower level for $\rho$ and $T$.} 
\label{tab MM3 fit} 
\begin{tabular}{ccccccccccccccccccc} 
\hline
Grid spacing &  & lin & log & log & log & log & log & lin & lin & log & log & lin & lin & log & log & lin & log\\ 
\hline 
Model & $\goodchi^2$ & $\rm A_V$ & d & $\rm R_*$ & $\rm T_*$ & $\rm m_{\rm disk}$  & $\rm r_{\rm disk}^{\rm max}$ & $\beta_{\rm disk}$ & $p_{\rm disk}$ & $h_{100}^{\rm disk}$ & $\rho_0^{\rm env}$ & $p_{\rm cav}$ & $\theta_0^{\rm cav}$ & $\rho_0^{\rm cav}$  & $r_{\rm disk}^{\rm min}$ & inc & $\rm L_*$\\ 
 &  & mag & kpc & ${\rm R}_\odot$ & K & ${\rm M}_\odot$ & au &  &  & au & g/cc &  & $\degr$  & g/cc & $R_{\rm sub}$ & $\degr$ & ${\rm L}_\odot$\\ 
\noalign{\smallskip} 
\hline 
\noalign{\smallskip} 
aer04B9J & $ 5842$ & $ 16.5$ & $ 6.75$ & $ 11.5$ & $ 15720$ & $ 0.0981$ & $ 440$ & $ 1.06$ & $ -0.205$ & $ 9.29$ & $ 7.28\,{\times}\, 10^{-21}$ & $ 1.16$ & $ 40.4$ & $ 1.09\,{\times}\, 10^{-23}$ & $ 3.35$ & $ 60.5$ & $ 7161$ \\ 
aer04B9J & $ 6799$ & $ 17.1$ & $ 6.75$ & $ 11.5$ & $ 15720$ & $ 0.0981$ & $ 440$ & $ 1.06$ & $ -0.205$ & $ 9.29$ & $ 7.28\,{\times}\, 10^{-21}$ & $ 1.16$ & $ 40.4$ & $ 1.09\,{\times}\, 10^{-23}$ & $ 3.35$ & $ 52$ & $ 7161$ \\ 
rlHDFVxN & $ 7251$ & $ 16.6$ & $ 6.07$ & $ 9.32$ & $ 15680$ & $ 0.0321$ & $ 1216$ & $ 1.13$ & $ -1.32$ & $ 11.8$ & $ 9.75\,{\times}\, 10^{-23}$ & $ 1.64$ & $ 32.8$ & $ 8.59\,{\times}\, 10^{-22}$ & $ 3.05$ & $ 59.6$ & $ 4639$ \\ 
aer04B9J & $ 7291$ & $ 17.3$ & $ 7.11$ & $ 11.5$ & $ 15720$ & $ 0.0981$ & $ 440$ & $ 1.06$ & $ -0.205$ & $ 9.29$ & $ 7.28\,{\times}\, 10^{-21}$ & $ 1.16$ & $ 40.4$ & $ 1.09\,{\times}\, 10^{-23}$ & $ 3.35$ & $ 39.5$ & $ 7161$ \\ 
aer04B9J & $ 7694$ & $ 17.5$ & $ 6.75$ & $ 11.5$ & $ 15720$ & $ 0.0981$ & $ 440$ & $ 1.06$ & $ -0.205$ & $ 9.29$ & $ 7.28\,{\times}\, 10^{-21}$ & $ 1.16$ & $ 40.4$ & $ 1.09\,{\times}\, 10^{-23}$ & $ 3.35$ & $ 42.3$ & $ 7161$ \\ 
Ipv9dWMM & $ 9668$ & $ 22.7$ & $ 6.4$ & $ 11.4$ & $ 16700$ & $ 0.0795$ & $ 244$ & $ 1.15$ & $ -0.161$ & $ 13.2$ & $ 2.19\,{\times}\, 10^{-24}$ & $ 1.79$ & $ 11.4$ & $ 1.17\,{\times}\, 10^{-22}$ & $ 1$ & $ 56.7$ & $ 8932$ \\ 
rlHDFVxN & $ 10702$ & $ 16.8$ & $ 6.07$ & $ 9.32$ & $ 15680$ & $ 0.0321$ & $ 1216$ & $ 1.13$ & $ -1.32$ & $ 11.8$ & $ 9.75\,{\times}\, 10^{-23}$ & $ 1.64$ & $ 32.8$ & $ 8.59\,{\times}\, 10^{-22}$ & $ 3.05$ & $ 47.8$ & $ 4639$ \\ 
WFmZWZb5 & $ 10838$ & $ 28.6$ & $ 6.07$ & $ 13.9$ & $ 13460$ & $ 0.0754$ & $ 207$ & $ 1.05$ & $ -0.924$ & $ 13.6$ & $ 2\,{\times}\, 10^{-22}$ & $ 1.23$ & $ 16.9$ & $ 1.53\,{\times}\, 10^{-22}$ & $ 1$ & $ 34.8$ & $ 5571$ \\ 
jZiuuzpK & $ 11033$ & $ 26.3$ & $ 7.11$ & $ 10.9$ & $ 25230$ & $ 0.0653$ & $ 1559$ & $ 1.15$ & $ -1.72$ & $ 4.11$ & $ 7.1\,{\times}\, 10^{-23}$ & $ 1.45$ & $ 14.4$ & $ 2.63\,{\times}\, 10^{-23}$ & $ 1$ & $ 55$ & $ 42775$ \\ 
MR8w9seO & $ 11686$ & $ 28$ & $ 6.07$ & $ 5.01$ & $ 23850$ & $ 0.0321$ & $ 287$ & $ 1.14$ & $ -1.42$ & $ 12.4$ & $ 3.7\,{\times}\, 10^{-22}$ & $ 1.01$ & $ 16.7$ & $ 4.14\,{\times}\, 10^{-21}$ & $ 1$ & $ 51.8$ & $ 7170$ \\ 
\noalign{\smallskip} 
\hline 
mean model \\
\noalign{\smallskip} 
\hline 
\noalign{\smallskip} 
mean & & $ 19.9$ & 6.53 & $ 10.5$ & $ 16672$ & $ 0.0679$ & $ 513$ & $ 1.09$ & $ -0.678$ & $ 9.91$ & $ 6.66\,{\times}\, 10^{-22}$ & $ 1.33$ & $ 30.9$ & $ 6.3\,{\times}\, 10^{-23}$ & $ 2.25$ & $ 50.6$ & $ 7541$ \\ 
sigma & &$ 4.28$ &$ 1.06$ & $ 1.26$ &$ 1.18$ &$ 1.6$ &$ 1.86$ &$ 0.0389$ &$ 0.554$ &$ 1.34$ &$ 13.8$ &$ 0.233$ &$ 10.7$ &$ 7.88$ &$ 1.73$ &$ 7.95$ &$ 1.71$ \\ 
\noalign{\smallskip} 
\hline 
\end{tabular} 
\end{sidewaystable}
\clearpage

\begin{sidewaystable}
\vspace*{9cm}
\caption[]{MM1 SED fit results: The table features the same structure as Table \ref{tab MM3 fit}, but additionally the burst-fit is included. %Since we use the same models for the burst-fit but with an increased source luminosity, the parameters (despite $r_*$) are the same. 
The burst names are composed of the original model-name and the adapted luminosity increase. Consequently the weighted mean and $\sigma$ are obtained taking into account both, the pre- and burst-results (see text). Only the parameters, which refers to the source are taken from the pre-burst alone (they change due to the burst). Again, the confidence interval in the case of log-sampled values extends from $\frac{x}{\sigma}$ to $x \,{\times}\, \sigma$.  %Since they are changing during the burst it doesn't make sense to give the mean values over both epochs. 
Note that the inner radius of the disk is not a free parameter (unlike for MM3). Instead, it is governed by the sublimation radius (at $T_{\rm sub}\,{=}\,1600$K). An outward shift of the sublimation radius may lower the dust masses for the post-/burst. 
% The third header line indicates whether the parameter space is linearly (lin) or logarithmically (log) sampled.}%In the table, we give only the disk dust-masses without sublimation.
} 
\label{tab MM1 fit} 
\begin{tabular}{ccccccccccccccccc}
\hline
Grid spacing &  & lin & log & log & log & log & log & lin & lin & log & log & lin & lin & log & lin & log\\
\hline 
Model & $\goodchi^2$ & av & d & $\rm R_*$ & $\rm T_*$ & $\rm m_{\rm disk}$ & $\rm r_{\rm disk}^{\rm max}$ & $\beta_{\rm disk}$ & $p_{\rm disk}$ & $h_{\rm 100}^{\rm disk}$ & $\rho_0^{\rm env}$ & $p_{\rm cav}$ & $\theta_0^{\rm cav}$ & $\rho_0^{\rm cav}$ & inc & $\rm L_*$\\ 
 &  & mag & kpc & $R_\odot$ & K & ${\rm M}_\odot$ & au &  &  & au & ${\rm g/cc}$ &  & $\degr$  & ${\rm g/cc}$ & $\degr$ & ${\rm L}_\odot$\\ 
\noalign{\smallskip} 
\hline 
Pre-burst\\
\noalign{\smallskip} 
\hline 
\noalign{\smallskip} 
eGYXcOh8\_02 & $ 42.8$ & $ 67.6$ & $ 7.11$ & $ 3.79$ & $ 25590$ & $ 0.000217$ & $ 3110$ & $ 1.18$ & $ -0.131$ & $ 2.34$ & $ 5.25\,{\times}\,10^{-19}$ & $ 1.56$ & $ 31.5$ & $ 1.81\,{\times}\,10^{-23}$ & $ 19.3$ & $ 5437$ \\ 
90Yt0exl\_03 & $ 46.6$ & $ 65.5$ & $ 6.4$ & $ 5.51$ & $ 19670$ & $ 0.0357$ & $ 1191$ & $ 1.13$ & $ -0.345$ & $ 1.25$ & $ 2.07\,{\times}\,10^{-18}$ & $ 1.39$ & $ 36.4$ & $ 1.1\,{\times}\,10^{-21}$ & $ 24.1$ & $ 4017$ \\ 
PUyhjE8Z\_02 & $ 47.5$ & $ 48.2$ & $ 6.75$ & $ 14.1$ & $ 12130$ & $ 4.01\,{\times}\,10^{-8}$ & $ 278$ & $ 1.28$ & $ -1.46$ & $ 11.4$ & $ 2.07\,{\times}\,10^{-17}$ & $ 1.76$ & $ 23.4$ & $ 2.71\,{\times}\,10^{-22}$ & $ 11.3$ & $ 3797$ \\ 
nTRTrE7X\_02 & $ 51.2$ & $ 59$ & $ 7.11$ & $ 2.71$ & $ 29889$ & $ 8.1\,{\times}\,10^{-6}$ & $ 423$ & $ 1.13$ & $ -0.607$ & $ 1.85$ & $ 9.58\,{\times}\,10^{-18}$ & $ 1.94$ & $ 32.7$ & $ 4.99\,{\times}\,10^{-21}$ & $ 18.1$ & $ 5180$ \\ 
nTRTrE7X\_03 & $ 55$ & $ 47.5$ & $ 6.75$ & $ 2.71$ & $ 29889$ & $ 8.1\,{\times}\,10^{-6}$ & $ 423$ & $ 1.13$ & $ -0.607$ & $ 1.85$ & $ 9.58\,{\times}\,10^{-18}$ & $ 1.94$ & $ 32.7$ & $ 4.99\,{\times}\,10^{-21}$ & $ 23.3$ & $ 5180$ \\ 
TJyUR9bA\_02 & $ 57.8$ & $ 62.6$ & $ 7.11$ & $ 3.91$ & $ 23840$ & $ 3.1\,{\times}\,10^{-7}$ & $ 631$ & $ 1.17$ & $ -0.102$ & $ 12.7$ & $ 7.15\,{\times}\,10^{-18}$ & $ 1.89$ & $ 46$ & $ 9.4\,{\times}\,10^{-23}$ & $ 14.4$ & $ 4366$ \\ 
4kW1TtMH\_04 & $ 59.1$ & $ 41.8$ & $ 6.4$ & $ 47.3$ & $ 7335$ & $ 0.000299$ & $ 140$ & $ 1.24$ & $ -0.752$ & $ 2.49$ & $ 5.14\,{\times}\,10^{-17}$ & $ 1.35$ & $ 32.9$ & $ 8.17\,{\times}\,10^{-22}$ & $ 30.6$ & $ 5715$ \\ 
lQ0YS3aF\_02 & $ 63.2$ & $ 70$ & $ 6.4$ & $ 6.7$ & $ 17980$ & $ 0.00527$ & $ 3798$ & $ 1.06$ & $ -0.775$ & $ 17.3$ & $ 3.24\,{\times}\,10^{-19}$ & $ 1.06$ & $ 12.5$ & $ 8.66\,{\times}\,10^{-22}$ & $ 13.5$ & $ 4150$ \\ 
qWHuujku\_05 & $ 64.4$ & $ 70$ & $ 6.75$ & $ 38$ & $ 8551$ & $ 2.63\,{\times}\,10^{-5}$ & $ 1015$ & $ 1.23$ & $ -1.62$ & $ 1.52$ & $ 2.33\,{\times}\,10^{-18}$ & $ 1.28$ & $ 47.8$ & $ 9.27\,{\times}\,10^{-21}$ & $ 46.8$ & $ 6832$ \\ 
qWHuujku\_06 & $ 65.2$ & $ 36.2$ & $ 6.75$ & $ 38$ & $ 8551$ & $ 2.63\,{\times}\,10^{-5}$ & $ 1015$ & $ 1.23$ & $ -1.62$ & $ 1.52$ & $ 2.33\,{\times}\,10^{-18}$ & $ 1.28$ & $ 47.8$ & $ 9.27\,{\times}\,10^{-21}$ & $ 50$ & $ 6832$ \\ 
\noalign{\smallskip} 
\hline 
Burst\\
\noalign{\smallskip} 
\hline 
\noalign{\smallskip} 
90Yt0exl\_L5.0 & $ 226$ & $ 65.5$ & $ 6.4$ & $ 12.3$ & $ 19670$ & $ 0.0357$ & $ 1191$ & $ 1.13$ & $ -0.345$ & $ 1.25$ & $ 2.07\,{\times}\,10^{-18}$ & $ 1.39$ & $ 36.4$ & $ 1.1\,{\times}\,10^{-21}$ & $ 24.1$ & $ 20078$ \\ 
90Yt0exl\_L5.5 & $ 229$ & $ 65.5$ & $ 6.4$ & $ 12.9$ & $ 19670$ & $ 0.0357$ & $ 1191$ & $ 1.13$ & $ -0.345$ & $ 1.25$ & $ 2.07\,{\times}\,10^{-18}$ & $ 1.39$ & $ 36.4$ & $ 1.1\,{\times}\,10^{-21}$ & $ 24.1$ & $ 22081$ \\ 
eGYXcOh8\_L4.5 & $ 234$ & $ 67.6$ & $ 7.11$ & $ 8.04$ & $ 25590$ & $ 0.000217$ & $ 3110$ & $ 1.18$ & $ -0.131$ & $ 2.34$ & $ 5.25\,{\times}\,10^{-19}$ & $ 1.56$ & $ 31.5$ & $ 1.81\,{\times}\,10^{-23}$ & $ 19.3$ & $ 24470$ \\ 
eGYXcOh8\_L5.0 & $ 234$ & $ 67.6$ & $ 7.11$ & $ 8.47$ & $ 25590$ & $ 0.000217$ & $ 3110$ & $ 1.18$ & $ -0.131$ & $ 2.34$ & $ 5.25\,{\times}\,10^{-19}$ & $ 1.56$ & $ 31.5$ & $ 1.81\,{\times}\,10^{-23}$ & $ 19.3$ & $ 27185$ \\ 
90Yt0exl\_L4.5 & $ 236$ & $ 65.5$ & $ 6.4$ & $ 11.7$ & $ 19670$ & $ 0.0357$ & $ 1191$ & $ 1.13$ & $ -0.345$ & $ 1.25$ & $ 2.07\,{\times}\,10^{-18}$ & $ 1.39$ & $ 36.4$ & $ 1.1\,{\times}\,10^{-21}$ & $ 24.1$ & $ 18077$ \\ 
TJyUR9bA\_L5.0 & $ 236$ & $ 62.6$ & $ 7.11$ & $ 8.74$ & $ 23840$ & $ 3.1\,{\times}\,10^{-7}$ & $ 631$ & $ 1.17$ & $ -0.102$ & $ 12.7$ & $ 7.15\,{\times}\,10^{-18}$ & $ 1.89$ & $ 46$ & $ 9.4\,{\times}\,10^{-23}$ & $ 14.4$ & $ 21828$ \\ 
TJyUR9bA\_L4.5 & $ 243$ & $ 62.6$ & $ 7.11$ & $ 8.3$ & $ 23840$ & $ 3.1\,{\times}\,10^{-7}$ & $ 631$ & $ 1.17$ & $ -0.102$ & $ 12.7$ & $ 7.15\,{\times}\,10^{-18}$ & $ 1.89$ & $ 46$ & $ 9.4\,{\times}\,10^{-23}$ & $ 14.4$ & $ 19644$ \\ 
eGYXcOh8\_L5.5 & $ 244$ & $ 67.6$ & $ 7.11$ & $ 8.88$ & $ 25590$ & $ 0.000217$ & $ 3110$ & $ 1.18$ & $ -0.131$ & $ 2.34$ & $ 5.25\,{\times}\,10^{-19}$ & $ 1.56$ & $ 31.5$ & $ 1.81\,{\times}\,10^{-23}$ & $ 19.3$ & $ 29907$ \\ 
90Yt0exl\_L6.0 & $ 245$ & $ 65.5$ & $ 6.4$ & $ 13.5$ & $ 19670$ & $ 0.0357$ & $ 1191$ & $ 1.13$ & $ -0.345$ & $ 1.25$ & $ 2.07\,{\times}\,10^{-18}$ & $ 1.39$ & $ 36.4$ & $ 1.1\,{\times}\,10^{-21}$ & $ 24.1$ & $ 24108$ \\ 
qWHuujku\_L4.5 & $ 245$ & $ 36.2$ & $ 6.75$ & $ 80.7$ & $ 8551$ & $ 2.63\,{\times}\,10^{-5}$ & $ 1015$ & $ 1.23$ & $ -1.62$ & $ 1.52$ & $ 2.33\,{\times}\,10^{-18}$ & $ 1.28$ & $ 47.8$ & $ 9.27\,{\times}\,10^{-21}$ & $ 50$ & $ 30745$ \\ 
\noalign{\smallskip} 
\hline 
Post-burst\\
\noalign{\smallskip} 
\hline 
\noalign{\smallskip} 
lQ0YS3aF\_L2.5 & $ 92.7$ & $ 70$ & $ 6.4$ & $ 10.6$ & $ 17980$ & $ 0.00527$ & $ 3798$ & $ 1.06$ & $ -0.775$ & $ 17.3$ & $ 3.24\,{\times}\,10^{-19}$ & $ 1.06$ & $ 12.5$ & $ 8.66\,{\times}\,10^{-22}$ & $ 13.5$ & $ 10376$ \\ 
nTRTrE7X\_L2.5 & $ 103$ & $ 59$ & $ 7.11$ & $ 4.29$ & $ 29889$ & $ 8.1\,{\times}\,10^{-6}$ & $ 423$ & $ 1.13$ & $ -0.607$ & $ 1.85$ & $ 9.58\,{\times}\,10^{-18}$ & $ 1.94$ & $ 32.7$ & $ 4.99\,{\times}\,10^{-21}$ & $ 18.1$ & $ 12950$ \\ 
lQ0YS3aF\_L3.0 & $ 105$ & $ 70$ & $ 6.4$ & $ 11.6$ & $ 17980$ & $ 0.00527$ & $ 3798$ & $ 1.06$ & $ -0.775$ & $ 17.3$ & $ 3.24\,{\times}\,10^{-19}$ & $ 1.06$ & $ 12.5$ & $ 8.66\,{\times}\,10^{-22}$ & $ 13.5$ & $ 12448$ \\ 
90Yt0exl\_L3.0 & $ 107$ & $ 65.5$ & $ 6.4$ & $ 9.54$ & $ 19670$ & $ 0.0357$ & $ 1191$ & $ 1.13$ & $ -0.345$ & $ 1.25$ & $ 2.07\,{\times}\,10^{-18}$ & $ 1.39$ & $ 36.4$ & $ 1.1\,{\times}\,10^{-21}$ & $ 24.1$ & $ 12051$ \\ 
TJyUR9bA\_L3.0 & $ 110$ & $ 62.6$ & $ 7.11$ & $ 6.77$ & $ 23840$ & $ 3.1\,{\times}\,10^{-7}$ & $ 631$ & $ 1.17$ & $ -0.102$ & $ 12.7$ & $ 7.15\,{\times}\,10^{-18}$ & $ 1.89$ & $ 46$ & $ 9.4\,{\times}\,10^{-23}$ & $ 14.4$ & $ 13097$ \\ 
90Yt0exl\_L2.5 & $ 110$ & $ 65.5$ & $ 6.4$ & $ 8.71$ & $ 19670$ & $ 0.0357$ & $ 1191$ & $ 1.13$ & $ -0.345$ & $ 1.25$ & $ 2.07\,{\times}\,10^{-18}$ & $ 1.39$ & $ 36.4$ & $ 1.1\,{\times}\,10^{-21}$ & $ 24.1$ & $ 10044$ \\ 
nTRTrE7X\_L3.0 & $ 114$ & $ 59$ & $ 7.11$ & $ 4.69$ & $ 29889$ & $ 8.1\,{\times}\,10^{-6}$ & $ 423$ & $ 1.13$ & $ -0.607$ & $ 1.85$ & $ 9.58\,{\times}\,10^{-18}$ & $ 1.94$ & $ 32.7$ & $ 4.99\,{\times}\,10^{-21}$ & $ 18.1$ & $ 15540$ \\ 
TJyUR9bA\_L3.5 & $ 114$ & $ 62.6$ & $ 7.11$ & $ 7.32$ & $ 23840$ & $ 3.1\,{\times}\,10^{-7}$ & $ 631$ & $ 1.17$ & $ -0.102$ & $ 12.7$ & $ 7.15\,{\times}\,10^{-18}$ & $ 1.89$ & $ 46$ & $ 9.4\,{\times}\,10^{-23}$ & $ 14.4$ & $ 15281$ \\ 
eGYXcOh8\_L2.5 & $ 116$ & $ 67.6$ & $ 7.11$ & $ 5.99$ & $ 25590$ & $ 0.000217$ & $ 3110$ & $ 1.18$ & $ -0.131$ & $ 2.34$ & $ 5.25\,{\times}\,10^{-19}$ & $ 1.56$ & $ 31.5$ & $ 1.81\,{\times}\,10^{-23}$ & $ 19.3$ & $ 13591$ \\ 
nTRTrE7X\_L2.0 & $ 118$ & $ 59$ & $ 7.11$ & $ 3.83$ & $ 29889$ & $ 8.1\,{\times}\,10^{-6}$ & $ 423$ & $ 1.13$ & $ -0.607$ & $ 1.85$ & $ 9.58\,{\times}\,10^{-18}$ & $ 1.94$ & $ 32.7$ & $ 4.99\,{\times}\,10^{-21}$ & $ 18.1$ & $ 10362$ \\ 
\noalign{\smallskip} 
\hline 
mean model \\
\noalign{\smallskip} 
\hline 
\noalign{\smallskip} 
%mean & $ $ & $ 48.9$ & 6.79 & $ 9.1$* & $ 16244$* & $ 4.62\,{\times}\, 10^{-5}$ & $ 958$ & $ 1.13$ & $ -0.98$ & $ 3.69$ & $ 2.81\,{\times}\, 10^{-18}$ & $ 1.49$ & $ 28.7$ & $ 7.44\,{\times}\, 10^{-22}$ & $ 21.7$ & $ 5090$* \\ 
%sigma & $ $ &$ 14.3$ &$ 1.05$ & $ 2.85$ &$ 1.66$ &$ 36.4$ &$ 2.45$ &$ 0.0859$ &$ 0.566$ &$ 2.59$ &$ 3.93$ &$ 0.284$ &$ 11$ &$ 7$ &$ 11.5$ &$ 1.20$ \\ 
mean & & $ 60.5$ & 6.77 & $ 8.38$* & $ 16834$* & $ 8.42\,{\times}\,10^{-5}$ & $ 952$ & $ 1.16$ & $ -0.592$ & $ 3.37$ & $ 2.98\,{\times}\,10^{-18}$ & $ 1.57$ & $ 33.7$ & $ 6.37\,{\times}\,10^{-22}$ & $ 21.8$ & $ 4984$* \\ 
sigma &&$ 9.7$ &$ 1.05$ & $ 2.86$ &$ 1.66$ &$ 72.2$ &$ 2.51$ &$ 0.06$ &$ 0.480$ &$ 2.72$ &$ 4.14$ &$ 0.31$ &$ 10.2 $ &$ 7.42$ &$ 10.1$ &$ 1.22$ \\ 
\noalign{\smallskip} 
\hline 
\footnotesize
\text{ * Pre-burst-value}
\end{tabular}
\end{sidewaystable}

\clearpage
\begin{table}
\begin{threeparttable}
\caption[]{Optical depth in the $V$-band for the 10 best MM1 pre-burst models along the mid-plane and along the line of sight toward the center. The intrinsic extinction dominates the total extinction (including the interstellar extinction) by far, as indicated by the huge values in the $V$-band.
%At $890\,\mu$m the system is optically thin in most of the cases, even in the disk midplane. 
For $\tau^{\rm mid-plane}$ the contribution of the disk is given separately. For some models $\tau_{\rm disk}$ almost covers the total extinction (along the line of sight), whereas for some models the contribution of the envelope (which is the same as $\tau_{\rm total}^{\rm mid-plane} - \tau_{\rm disk}^{\rm mid-plane}$) is the dominant one. Note that the optical depths are given for the pre-burst, during the burst the optical depths may be slightly lower because of dust sublimation.}
\label{tab: tau_V}
\begin{tabular}{ccccc}
\hline
\noalign{\smallskip}
Model & inc [\degr]& $\tau_{\rm disk,\, V}^{\rm mid-plane}$ & $\tau_{\rm total,\, V}^{\rm mid-plane}$ & $\tau_{\rm total,\, V}^{\rm line\,of\,sight}$ \\
\noalign{\smallskip}
\hline
\noalign{\smallskip}
eGYXcOh8\_{02} & 19.3 & 360 & 3200 & 380 \\ 
90Yt0exl\_{03} &24.1& $1.2\,{\times}\,10^6$ & $1.2\,{\times}\,10^6$ & 390\\
PUyhjE8Z\_{02} &11.3& 30 & 11000 & 360\\ 
nTRTrE7X\_{02} & 18.1& 1700 & 9300 & 420\\ 
nTRTrE7X\_{03} &23.3& 1700 & 9300 & 510\\ 
TJyUR9bA\_{02} & 14.5& 2 & 11000 & 240\\
4kW1TtMH\_{04} &30.6&$3.4\,{\times}\,10^5$ & $3.5\,{\times}\,10^5$ & 830\\ 
lQ0YS3aF\_{02} &13.5&$1.2\,{\times}\,10^5$ &$1.4\,{\times}\,10^6$ & 889\\ 
qWHuujku\_{05} &46.8&$6.2\,{\times}\,10^5$ &$6.6\,{\times}\,10^5$ & 920\\ 
qWHuujku\_{06} &50.0 &$6.2\,{\times}\,10^5$ &$6.6\,{\times}\,10^5$ & 1200\\ 
%vL7XftIX & 11.5&2700 & 6600 & 490 \\ 
\noalign{\smallskip}
\hline
%{\red\tt Include Meanmodel}\\
mean model & 21.8 & 3300 & 11000 & 430\\
\noalign{\smallskip}
\hline
\end{tabular}
\end{threeparttable}
\end{table}

\end{appendix}
%sed_eGYXcOh8_02.out 19.287005775111844 0.004887331390831092 0.10173452781690456
%sed_90Yt0exl_03.out 24.10984976951866 0.004942527485645818 14.511528312107984
%sed_PUyhjE8Z_02.out 11.345391310441666 0.004467384628342461 0.3176180662989397
%sed_nTRTrE7X_02.out 18.0974993592283 0.0051551224783524235 0.25018568682504805
%sed_nTRTrE7X_03.out 23.325014890875597 0.006363388842395944 0.25018568682504805
%sed_TJyUR9bA_02.out 14.451744073403137 0.0029859327231221925 0.6768830535192667
%sed_4kW1TtMH_04.out 30.58847197675169 0.010349914150128548 4.487812984835409
%sed_lQ0YS3aF_02.out 13.50822690285383 0.011291089851971052 0.21438752583122436
%sed_qWHuujku_05.out 46.75229512487008 0.011482906309330005 0.9668699976904938
%ed_qWHuujku_06.out 50.0418239631957 0.014909022471582151 0.9668699976904938

\end{document}

% last table with 890um values
\begin{table}
\begin{threeparttable}
\caption[]{Optical depth in the $V$-band and at $890\mu m$ for the 10 best MM1 pre-burst models along the mid-plane and along the line of sight toward the center. The intrinsic extinction dominates the total extinction (including the interstellar extinction) by far, as indicated by the huge values in the $V$-band. At $890\,\mu$m the system is optically thin in most of the cases, even in the disk midplane. For $\tau^{\rm mid-plane}$ the contribution of the disk is given separately. For some models $\tau_{\rm disk}$ almost covers the total extinction (along the line of sight), whereas for some models the contribution of the envelope (which is the same as $\tau_{\rm total}^{\rm mid-plane} - \tau_{\rm disk}^{\rm mid-plane}$) is the dominant one. Note that the optical depths are given for the pre-burst, during the burst the optical depths may be slightly lower because of dust sublimation.}
\label{tab: tau_V}
\begin{tabular}{cccccccc}
\hline
\noalign{\smallskip}
Model & inc [\degr]& $\tau_{\rm disk,\, V}^{\rm mid-plane}$ & $\tau_{\rm total,\, V}^{\rm mid-plane}$ & $\tau_{\rm total,\, V}^{\rm line\,of\,sight}$ 
& $\tau_{\rm disk,\, 890\mu m}^{\rm mid-plane}$ & $\tau_{\rm total,\, 890\mu m}^{\rm mid-plane}$ & $\tau_{\rm total,\, 890\mu m}^{\rm line\,of\,sight}$\\
\noalign{\smallskip}
\hline
\noalign{\smallskip}
eGYXcOh8\_{02} & 19.3 & 360 & 3200 & 380 & $4\,{\times}\,10^{-3}$ & 0.04 & $5\,{\times}\,10^{-3}$\\ 
90Yt0exl\_{03} &24.1& $1.2\,{\times}\,10^6$ & $1.2\,{\times}\,10^6$ & 390 & 15.1 & 15.2 &$5\,{\times}\,10^{-3}$\\
PUyhjE8Z\_{02} &11.3& 30 & 11000 & 360 & $4\,{\times}\,10^{-4}$ & 0.1 & $4\,{\times}\,10^{-3}$\\ 
nTRTrE7X\_{02} & 18.1& 1700 & 9300 & 420 & 0.02 & 0.13 &$5\,{\times}\,10^{-3}$\\ 
nTRTrE7X\_{03} &23.3& 1700 & 9300 & 510 & 0.02 & 0.11 &$6\,{\times}\,10^{-3}$\\ 
TJyUR9bA\_{02} & 14.5& 2 & 11000 & 240 &  $2\,{\times}\,10^{-5}$ & 0.11&$3\,{\times}\,10^{-3}$\\
4kW1TtMH\_{04} &30.6&$3.4\,{\times}\,10^5$ & $3.5\,{\times}\,10^5$ & 830 & 4.2 & 4.3& 0.010\\ 
lQ0YS3aF\_{02} &13.5&$1.2\,{\times}\,10^5$ &$1.4\,{\times}\,10^6$ & 889 & 0.15&0.18& 0.011\\ 
qWHuujku\_{05} &46.8&$6.2\,{\times}\,10^5$ &$6.6\,{\times}\,10^5$ & 920 & 0.77 & 0.82& 0.011\\ 
qWHuujku\_{06} &50.0 &$6.2\,{\times}\,10^5$ &$6.6\,{\times}\,10^5$ & 1200 & 0.77 & 0.82 & 0.015\\ 
%vL7XftIX & 11.5&2700 & 6600 & 490 \\ 
\noalign{\smallskip}
\hline
%{\red\tt Include Meanmodel}\\
mean model & 21.8 & 3300 & 11000 & 430 & 0.04 & 0.13 & $5\,{\times}\,10^{-3}$\\
\noalign{\smallskip}
\hline
\end{tabular}
\end{threeparttable}
\end{table}

